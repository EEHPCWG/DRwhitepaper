Some of the largest supercomputing centers in the United States are developing 
new relationships with their electricity service providers. 
These relationships, similar to other commercial and industrial partnerships, are 
driven by mutual interest to reduce energy costs and improve electrical grid reliability.
While supercomputing centers are concerned about electricity quality, environmental 
impact and availability, electricity service providers are concerned about impacts
on electrical grid reliability, particularly in terms of energy consumption, peak power and power fluctuations.
Supercomputing centers' power demand can be 20 MW or more---the 
theoretical peak power requirements are greater than 45 MW---and recurring 
intra-hour variability can exceed 8 MW.
This paper evaluates today`s relationships, potential partnerships and possible 
integration between supercomputing centers and their electricity service providers.
The paper uses feedback from a questionnaire submitted to supercomputing centers 
on the Top100 List in the United States to describe opportunities for overcoming the 
challenges to HPC-Grid integration.

