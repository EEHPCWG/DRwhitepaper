Some of the largest supercomputer centers in the United States are developing 
new relationships with their electricity service providers. 
These relationships, similar to other commercial and industrial facilities, are 
driven by mutual interest to save energy costs and improve electrical grid reliability.
Supercomputer centers are concerned about price, quality, environmental 
impact and availability.
Electricity service providers are concerned about the supercomputer center`s impact 
on the electrical grid reliability in terms of energy consumption, peak power and fluctuations in power.
Supercomputer center power demand can be greater than 20MW or more---the 
theoretical peak power requirements are greater than 45MW---and re-occurring 
intra-hour variability can exceed 8MW.
Consequently, the electricity service providers for some supercomputer centers
are asking for hourly forecasts of power demand, a day in advance.
This paper evaluates today`s relationships, potential partnerships and possible 
integration between supercomputer centers and their electricity service providers.
The paper uses feedback from a questionnaire submitted to supercomputer centers 
on the Top100 List in the United States to list opportunities for overcoming the 
challenges to HPC-Grid integration.

