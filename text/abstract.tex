Some of the largest supercomputer centers in the United States are developing 
new relationships with their electricity service providers. 
These relationships, similar to other commercial and industrial partnerships, are 
driven by mutual interest to reduce energy costs and improve electrical grid reliability.
Supercomputer centers are concerned about electricity quality, environmental 
impact and availability,
while electricity service providers are concerned about impacts
on electrical grid reliability, particularly in terms of energy consumption, peak power and power fluctuations.
Supercomputer centers' power demand can be 20MW or more---the 
theoretical peak power requirements are greater than 45MW---and recurring 
intra-hour variability can exceed 8MW.
%Consequently, the electricity service providers for some supercomputer centers
%are asking for hourly forecasts of power demand, a day in advance.
This paper evaluates today`s relationships, potential partnerships and possible 
integration between supercomputer centers and their electricity service providers.
The paper uses feedback from a questionnaire submitted to supercomputer centers 
on the Top100 List in the United States to list opportunities for overcoming the 
challenges to HPC-Grid integration.

