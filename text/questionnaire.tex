We used a questionnaire to understand the current experiences of a
supercomputer center's interaction with their electricity providers. We restricted the analysis to sites in the United States
because the results of the survey and practices of demand response is highly
correlated and driven by energy policies in the country. 
\cite{torriti_demand_2010}
[Tor10].

Nineteen Top100 List sized sites in the United States were targeted for the
questionnaire. Eleven sites responded (Oak Ridge National Laboratory, 
Lawrence Livermore National Laboratory, 
Argonne National Laboratory, 
Los Alamos National Laboratory, LBNL, 
Wright Patterson Air Force Base,
National Oceanic Atmospheric Administration (NOAA), 
National Center for Supercomputing Applications, 
San Diego Supercomputing Center (SDSC), 
Purdue University and Intel Corporation). The questionnaire was
sent to a sample that was not randomly selected. It was sent to those sites
where it was relatively easy to identify an individual based on membership
within the EE HPC WG. The sample is more representative of Top50 sized sites
(1 Top50 sized site was not in the sample and 60{\%} (9/15) of the sample
responded). Only 4 additional sites were sampled from the Top51-Top100 List
and, of those, 2 responded (Intel and National Oceanic and Atmospheric Administration).

The total power load as well as the intra-hour fluctuation of these sites
varied significantly. Total power load includes all computing systems plus 
ancillary systems such as power delivery and cooling components.
There were four sites with total power load greater
than 10MW, two sites with \textasciitilde 5MW total power load and five
sites with less than 2MW total power load. For those with total power load greater than 10MW, the
intra-hour fluctuation varied from less than 3MW to 8MW. One of
\textasciitilde 5MW sites said that they experienced 4MW variability. We chose less than 3MW intra-hour
variability as the bottom of the scale because we assumed that the
electricity providers would not be affected by 3MW (or less) 
fluctuations. The
rest of the sites all reported less than 3MW intra-hour fluctuation. Most of the intra-hour variability
was due to preventative maintenance (again, the power variation includes both computing and ancillary systems).

%Patki: New Graph instead of Table 1

\begin{figure*}
\begin{center}
\includegraphics[scale=0.7]{NewGraphs/Table1-Graph.pdf}
\end{center}
\end{figure*}

%\begin{table}[htbp]
%
%\begin{center}
%\caption{Caption Number 1}
%\begin{tabular}{|p{65pt}|l|l|}
%\hline
%\textbf{Total Load}&
%\textbf{Variability}&
%\textbf{Frequency} \\
%\hline
%16-17MW&
%5MW&
%weekly \\
%\hline
%13-14MW&
%8MW&
%monthly \\
%\hline
%10-11MW&
%Less than 3MW&
%weekly \\
%\hline
%10-11MW&
%7MW&
%weekly \\
%\hline
%4-5MW&
%Less than 3MW&
%weekly \\
%\hline
%4-5MW&
%4MW&
%weekly \\
%\hline
%1-2MW&
%Less than 3MW&
%weekly \\
%\hline
%1-2MW&
%140kW&
%daily \\
%\hline
%1-2MW&
%Less than 3MW&
%yearly \\
%\hline
%1-2MW&
%200kW or less&
%daily \\
%\hline
%1-2MW&
%Less than 3MW&
%daily \\
%\hline
%\end{tabular}
%\label{tab1}
%\end{center}
%\end{table}

We asked if the supercomputer centers had talked to their electricity
providers about programs and methods used to balance the grid supply and
demand of electricity. About half of them have had some discussion, but it
has mostly been limited to programs (e.g., peak shed, dynamic pricing) 
and not methods (e.g., regulation, frequency response, congestion).


\begin{table}[htbp]

\begin{center}
\caption{Caption Number 2}
\begin{tabular}{|p{230pt}|l|}
\hline
\textbf{Discussions with Energy Providers}&
{\%} Answered Yes \\
\hline
\textbf{Demand-side programs}&
~ \\
\hline
Shedding load during peak demand&
54 \\
\hline
Responding to pricing incentive programs&
45 \\
\hline
Shifting load during peak demand&
36 \\
\hline
\textbf{Supply-side programs}&
~ \\
\hline
Enabling use of renewables&
36 \\
\hline
Congestion, Regulation, Frequency Response&
18 \\
\hline
Contributing to electrical grid storage&
10 \\
\hline
\end{tabular}
\label{tab2}
\end{center}
\end{table}

Approximately half of the respondents are not currently interested in shedding
load during peak demand. LANL reports that the "technical feasibility" and "business case has yet to be developed."
There is slightly more interest in shifting than shedding load. SDSC reports that
 ``Automatic load shedding is being
explored/deployed today'' for the entire campus, not just the supercomputing center.

Responding to pricing incentive programs is also not considered currently interesting to approximately half of the repondents,
although the reasons for this low interest may be organizational. Several
open-ended comments revealed that pricing is fixed and/or done by another
organization at the site level and outside of their immediate control.

Eighty percent of the respondents have not had discussions with their
electricity providers about congestion, regulation and frequency
response. Los Alamos National Laboratory (LANL) is one of two who have had discussions and who commented that
they are ``learning about the process'' and that it is ``outside of [their]

visibility or control''.

There were many more respondents who have had discussions with their
electricity providers about enabling the use of renewables; 36{\%}
have already had discussions and more than half are interested in further
and/or future discussions. SDSC already has a site-wide program; ``the
campus has a large fuel cell (2.5$+$ MW) and works with the utility with
renewables.'' Other responses suggest that the interest is at the site level
and not unique to the supercomputer center.

An open-ended question was posed as to whether or not there was information
either requested of the supercomputer sites by their providers or,
conversely, requested of the providers by the sites. In both cases, well
over 75{\%} of the respondents answered no. Lawrence Livermore National Laboratory (LLNL) and LANL were the
exceptions. LLNL is ``responding to requests for additional data on an hourly, 
weekly and monthly basis."  They are also working to develop an automated capability to share 
data with their electricity providers, which would provide automated additional 
detailed forecasting and ultimately real time data."
LANL has also been requested to provide ``power projections, hour by hour,
for at least a day in advance'' and, perhaps as a consequence, would like to
have more information on ``sensitivity of power distribution grid to rapid
transients (random daily step changes of 10 MW up or down within a single AC
cycle).''

Given the low levels of current engagement between the electricity
providers and the supercomputer centers, it is not surprising that none of
the supercomputer centers are currently using any power management
strategies to respond to grid requests by their energy service
providers. SDSC's \textit{supercomputer center} is not an exception, but they did respond that their
entire ``campus is leveraging parallel electrical distribution to trigger
diesel generators and other back-up resources to respond to grid and
non-grid requests.''

It was suggested by ORNL that some of the power management strategies 
are of questionable business value even for energy efficiency, let alone grid integration.
For example, ORNL comments that "these assets have very clear depreciation schedules, and the modest cost 
savings in terms of electricity consumption due to some of these methods may not (or frequently will not) 
outweigh the capital investment cost in the computer.  I.e. If a site spent \$100M for a computer that will 
remain in production for 60 months, then the 
apparent benefit of power capping, etc can easily be outweighed by lost productivity of the consumable resource.

Similarly, another comment by ORNL suggested that the rapid deployment of hardware features, like P-states,  
 may outpace the need for strategies like power aware job scheduling.

We tried to evaluate if power management strategies will be considered
relevant and effective for grid integration at some point in the future. Two
questions were asked: is there interest in using the strategies and what
impact did they think that the strategies would have? When combining
interest and impact, the results showed that power capping, shutdown, and
job scheduling were both potentially interesting and of high impact. 


Load migration, back-up
scheduling, fine-grained power management and thermal management were of medium
interest and impact. Lighting control and back-up resources were of low
interest and impact.  

Temperature control and lighting management are utilized as strategies, but considered medium to low interest and impact
for responding to requests from electricity service providers because they are used as strategies for 
energy efficiency rather than grid response.
The infrastructure energy efficiency of the responding supercomputer sites is high, as reflected in their reported
Power Usage Effectiveness (PUE).  Two sites reported a PUE below 1.25, the majority were between 
1.25 and 1.5 and the highest was 1.53. Approximately half of the respondents said that they used 
temperature control and lighting management
as strategies, but not for grid requests.  Temperature control and lighting management are well documented and understood
strategies for improving energy efficiency, so it is not surprising that sites with PUEs below 1.5 are using them.
NOAA comments that their "lights automatically shut off 24x7 
when there is no motion in the data center."  There is a value in lighting control for 
energy efficiency purposes, as demonstrated by its having been fully implemented.  NOAA also comments that
the impact of further lighting control "is so small compared to the HPC demand load that" they would "be surprised 
if the utility is interested."  


Distinguishing interest from impact sheds further insight; some strategies are 
considered high impact, but not interesting enough to consider deployment.  Facility
shutdown is rated as having a high impact, but only
considered interesting by 36\% of the respondents.  NOAA commented that, 
"We've had too many instability and equipment failures to utilize this as a strategy."  This divide is even more
apparent with load migration.  It is rated as having a high impact by 36\% of the
respondents, but only interesting to 10\% .  


\begin{table}[htbp]
\begin{center}
\caption{Summary of aspects and quality levels}
\begin{tabular}{|p{2.5in}|p{0.75in}|p{0.75in}|p{0.75in}|} \hline

\textbf{HPC strategies for responding to} &
\textbf{\%} &
\textbf{\%} &
\textbf{\%} \\

\textbf{Electricity Provider requests} &
\textbf{Interested} &
\textbf{High} &
\textbf{Medium} \\

\textbf{(listed from highest to lowest interest + impact)} &
 &
\textbf{Impact} & 
\textbf{Impact} \\
\hline

Coarse grained power management &
64 &
46 &
27 \\

Load migration &
10 &
36 &
18 \\
\hline

Re-scheduling back-ups &
45 &
0 &
10 \\
\hline

Fine-grained power management &
27 &
0 &
36 \\
\hline

Temperature control beyond ASHRAE limits &
27 &
0 &
18 \\
\hline

Turn off lighting &
18 &
0 &
0 \\
\hline

Use back-up resources (e.g., generators) &
0 &
10 &
27 \\
\hline

\end{tabular}
\label{tab3}
\end{center}

\end{table}


%\begin{table}[htbp]
%
%\begin{center}
%\caption{Caption Number 3}
%\begin{tabular}{|p{299pt}|}
%\hline
%\textbf{HPC strategies for responding to Energy Provider requests (listed from highest to lowest interest }$+$\textbf{ impact)} \\
%\hline
%Coarse grained power management \\
%\hline
%Facility shutdown \\
%\hline
%Job scheduling \\
%\hline
%Load migration \\
%\hline
%Re-scheduling back-ups \\
%\hline
%Fine grained power management \\
%\hline
%Temperature control beyond ASHRAE limits \\
%\hline
%Turn off lighting \\
%\hline
%Use back-up resources (e.g., generators) \\
%\hline
%\end{tabular}
%\label{tab3}
%\end{center}

%\end{table}
