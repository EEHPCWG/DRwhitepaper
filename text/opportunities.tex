The responses to the questionnaire presented in
Section~\ref{sec:questionaire} represent a variety of desires and
experience regarding interactions between supercomputing centers and
energy service providers.  For example, the responses from the two centers
with the largest power draws, Lawrence Livermore National Laboratory
(LLNL) and Oak Ridge National Laboratory (ORNL), diverge in several
areas.  This divergence is perhaps primarily due to characteristics of
their respective energy service providers.  In contrast, San Diego
Supercomputer Center (SDSC) stands out as a leader in integrating
with their energy service provider on a site-wide level.  To that end, the
responses from SDSC may exemplify some of the opportunities available
to other supercomputing centers that are willing to pursue this degree
of integration.

% The following paragraph is what we want to claim, but is actually
% factually incorrect.  A supercomputing center understands its
% cost to operate per unit time (e.g., cost to run per hour).  This
% cost includes things like the cost of personnel, cost of renting
% the physical space for the datacenter, cost of electricity, etc.
% It turns out that the cost of electricity is a small percent of
% the overall cost of running the center.  In the end, I think there
% is very little chance of getting supercomputing centers to
% willingly negotiate with energy service providers because the cost of
% the lost opportunity is too high since the remaining costs are
% unchanged.
The responses to the questionaire also suggest that some energy service
providers are requesting that their supercomputing center customers
develop capabilities for informing the provider of expected periods of
exceptional power consumption and for responding to requests from the
provider to consume less power for specified periods of time.  Upon
initial consideration, this idea might seem to run counter to the
primary mission objective of most supercomputing centers of delivering
as many uninterrupted computational cycles as possible to their users.
In some extreme cases, supercomputing centers may not have a choice
in the matter as the size and energy requirements of supercomputers
increase; indeed, some energy service providers may \textit{require} large
centers to develop a demand-response capability.  However, a direct
business case may exist to encourage supercomputing centers to develop
this negotiation capability on their own.  For example, if energy service
providers were to offer electricity at a significantly reduced rate
on the condition that the supercomputing center customer develop
demand-response capabilities, the long-term cost savings to the
center could make undertaking such a project worthwhile.

Perhaps one of the most straightforward ways that supercomputing
centers can begin the process of developing a demand-response
capability is by enhancing existing system software used for managing
computing resources within the center.  Indeed, the questionaire
responses from Section~\ref{sec:questionaire} as well as the literature
review presented in Section~\ref{sec:priorwork} both strongly support the
idea that the greatest opportunities for supercomputing centers to
develop integration capabilities are related to system software.
Specifically, and presented in approximate order of decreasing
interest and expected impact to the questionaire respondents, system
software in this context consists of coarse-grained power management
in the form of power capping, job scheduling, load migration,
rescheduling backups, and fine-grained power management.  Of these,
job scheduling may be a practical starting point simply because of the
unique role that the job scheduler and resource manager play within a
datacenter.

% I (koenig) would be able to self-cite out of my own publications
% concepts such as ``priority'' and ``urgency'' described in the
% following paragraph.  If desired, we would need to add these
% papers into the bibliography and/or related works section.
On one hand, the job scheduler has knowledge of and control over the
upcoming workflow within the supercomputing center simply by examining
and manipulating the job queue.  For example, jobs may be submitted
with various metadata that enable the job scheduler to understand
characteristics of each job such as \textit{priority}, the relative
importance of a job compared to other jobs, and \textit{urgency}, the
rate at which the value of a job decreases as time elapses.  These
characteristics are not only important to a job scheduler for ensuring
efficient utilization of a supercomputing center's resources under
traditional circumstances, but they are also a vital piece of
successfully implementing a demand-response capability for at least
two reasons.  First, they provide a set of metrics by which the
supercomputing center can estimate the cost in terms of the ``lost
opportunity'' of responding to an energy service provider's request to
run with attenuated resources.  Second, they allow the supercomputing
center to prioritize jobs in the queued workflow in order to understand
how to best utilize computational resources.  This capability is
important under normal circumstances, but becomes even more essential
in a demand-response scenario.

On the other hand, the job scheduler has knowledge of and control over
the computational resources within the supercomputing center, giving
the job scheduler several mechanisms for implementing a
demand-response capability.  Most of these mechanisms could be
considered fine-grained power control mechanisms because they mostly
tune low-level settings on the nodes and processors within the
supercomputing center.  For example, the job scheduler knows which
nodes within a supercomputer are occupied with running jobs or are
expected to become occupied in the near future.  To that end, the job
scheduler can use its control over the resource management process to
place idle nodes into a sleep state in which they draw significantly
reduced power.  This strategy is especially effective in
supercomputing environments containing at least some resources that
are used at irregular intervals, allowing opportunities to utilize
sleep states effectively during periods when the resources are idle.
In environments where all computing resources are heavily utilized
most of the time, more sophisticated strategies that require the job
scheduler to rely on knowledge about each batch job may be necessary.
Such knowledge might come from the type of metadata described in the
previous paragraph or from a database that is maintained based on
previous runs of jobs submitted by each user.  For example, if the job
scheduler knows that a given job contains mostly I/O operations, or
consists of discrete phases where I/O occurs, the job scheduler might
choose to adjust the Performance State (P-state) for each processor
running the job in a way that reduces the job's overall power
consumption.  The P-state mechanism is a way of scalably adjusting a
CPU's frequency and voltage operating points which in turn causes the
processor to consume less power directly and to produce less thermal
load indirectly.  In cases where a processor is executing a
processing-intense task, gating the processor's P-state often has a
noticable impact on the overall task performance; however, in cases
where a processor is executing a mostly I/O-bound task, gating the
processor's P-state typically does not make a noticable impact on the
overall task performance due to the fact that the processor spends a
great deal of time blocked waiting for I/O operations to complete.

Even more interesting scenarios are possible in cases where the job
scheduler combines its knowledge of the upcoming queued workflow with
its knowledge and control over the computational resources within
the supercomputing center.  These scenarios are most appropriate when
the supercomputing scenario contains a pervasively heterogeneous mix
of computational resources.  For example, many contemporary datacenters
contain several different types of compute nodes with various types of
processors and accelerator cards.  In some circumstances, the job
scheduler may be able to choose which resource to use for running a
given job among several candidate resources.  The trade-off here is
not only in terms of the time necessary to complete the job (i.e.,
different resources could potentially complete the job in very different
amounts of time) but also in terms of the energy consumed in completing
the job (i.e., different resources could potentially consume very
different amounts of energy in completing the job).  By maintaining a
database of job-to-resource mappings that record the time and energy
taken for each job, the scheduler can, over time, improve its ability
to decide which jobs have the highest affinity to each type of resource.
Using this knowledge to optimize a supercomputing center's workflow in
terms of job throughput or energy consumption is admittedly complex,
but the potential rewards are likely to be compelling both to the
day-to-day operation of the center and to demand-response capabilities.

% It may be possible to self-cite some of my upcoming publications
% that discuss how to do power capping in a scheduler.

% Comment from Ghaleb:
% We can argue the need for instrumentation not necessarily assume that they will exist. 
% Since this paper is encouraging SC centers to consider D/R then I think it makes sense to 
% encourage work in having better and finer resolution instrumentation capabilities.
Several of the ideas described in the preceeding paragraphs assume
that the supercomputing center environment has some amount of
instrumentation and metering that allows for the collection of power
telemetry data.  Not only is this telemetry necessary for the job
scheduler to be able to make decisions about the workflow and
resources it is to schedule, the telemetry is also important to the
datacenter facility manager in order to understand how the power
supplied by the energy service provider is distributed to resources
within the center.  In light of the fact that many system integrators
such as Cray and IBM are now delivering supercomputing systems that
include telemetry capabilities, the assumption that this information
is available seems acceptable.  According to the responses to the
questionnaire presented in the previous Section, datacenter facility
managers perceive this accounting data as distinct from the per-user
or per-job accounting data described above and indicate that this
data should be retained for electricity provisioning planning purposes.
At a very high level, this detailed knowledge of where electricity is
being used in a supercomputing center is an important piece in
capabilities such as power capping, where the overall consumption of
power is maintained at or below some maximum threshhold level.  Power
capping may be accomplished either manually by the facility manager
or automatically by the job scheduler, but both approaches require
detailed knowledge that come from monitoring and accounting.

Opportunities may also exist for supercomputing centers to cooperate
with each other in scenarios in which computational loads are migrated
from one site to another where energy costs are less expensive.  This
scenario is challenging for both technical reasons and business reasons.
Technical challenges include issues such as user authentication and
autorization (i.e., a user may be authorized to use resources at one
site but not at another site) and data movement (i.e., it may be
infeasible to migrate large datasets from one site to another site).
Business challenges include the notion that a supercomputing center
currently has little incentive to migrate jobs to another ``competing''
center.  Indeed, the questionnaire results reflect low interest in
load migration strategies.  It seems likely that in order to be a
feasible scenario, the structure of payment and rewards to a
supercomputing center to cooperate with other centers would need to
be structured differently than they are currently.

In a very broad sense, demand-response techniques such as job scheduling,
power capping, and load migration can be considered to be coarse-grained
approaches because they involve considering ``big picture'' views of the
workload and computational resources in a supercomputing center.  According
to the questionnaire results presented in the previous Section, facilities
managers view these approaches as the most likely candidates for creating
effective demand-response capabilities.

Finally, this Section has focused heavily on the opportunities available
to supercomputing centers that come from developing demand-response
capabilities.  This notion is primarily due to the fact that the
questionnaire presented in Section~\ref{sec:questionaire} was
distributed to high-performance computing centers in the United
States, not to energy service providers.  That said, opportunities do
exist for energy service providers that develop demand-response
capabilities.  At one level, the negotiation process itself requires
integration in terms of the communication and messaging protocols that
are necessary.  To that end, opportunities exist for adapting and extending
existing standards currently used within the industry, thus creating new
use cases and capabilities for energy service providers.  At a higher
level, energy service providers will most likely need to improve their
ability to determine in near real time the important places within the
electrical grid where demands exceed supply.  Determining this is likely
to be a complex optimization problem.  While this Section focuses on
solving these problems to the end of developing a demand-response
strategy in conjunction with supercomputing centers, these capabilities
are likely applicable to a broad range of customers.
