
[koenig -- 01-NOV-2013] These are things I think this section should
include:

\begin{enumerate}
\item Link the beginning of this section to the end of the previous (Survey Results) section by suggesting that the biggest ``opportuni
ty'' is to start a process of negotiation/interaction between utility providers and HPC centers; the survey data seems to indicate that
 this is being asked for by providers, at least in some small way already
\item opportunities related to system software
        \begin{itemize}
        \item If negotiation starts happening between utility providers and HPC center operators, the system software (i.e., job schedu
ler) is a key component in order to ensure that this happens as efficiently as possible in order to keep high utilization / business ut
ility going at the HPC center (consider here things such as fluctuations in HPC use; e.g., things like large-scale acceptance / Top500
style runs)
        \item just lowering the power consumption of a batch job (i.e., by using fine grained power management techniques) does not ens
ure that the overall energy consumption is reduced; need some kind of knowledge about the workflow in the organization to make these ki
nds of determinations
        \item in addition to the advanced reservation capabilities discussed in the three paragraphs above, other areas where system so
ftware can participate are
                \begin{itemize}
                \item power capping
                \item temperature adjustment within the datacenter (make reference to discussion on Page 6)
                \item load migration - requires cooperation by HPC centers; also, migrating large datasets is hard
                \end{itemize}
        \item need to expose ``knobs'' into the system software so that HPC facility managers can easily adjust the objectives that the
 system software is using to make decisions because the overall number of ways of scheduling a workflow makes the problem too hard to r
eadily solve by hand
        \end{itemize}
\item if there is some kind of automated ``negotiation'' process that takes place between utility providers and HPC centers, it's likel
y that the utility providers will need to improve their capabilities to be able to participate in this process (e.g., probably need to
solve some kind of weighted optimization problem in near real time in order to know where the most important places are to ensure unint
errupted service); there is an opportunity to the utility provider in this, however, in that these advancements in their technology for
 monitoring and adjusting their infrastructure might be leveraged toward other ends that are not related to HPC centers specifically
\end{enumerate}
