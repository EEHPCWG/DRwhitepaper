The responses to the questionnaire presented in
Section~\ref{sec:questionnaire} represent a variety of desires and
experience regarding interactions between SCs and
ESPs. For example, the responses from the two SCs
with the largest power draws, LLNL and ORNL, diverge in several
areas. This divergence is perhaps primarily due to characteristics of
their respective ESPs. In contrast, SDSC stands out as a leader in integrating
with their ESP on a site-wide level. To that end, the
responses from SDSC may exemplify some of the opportunities available
to other SCs that are willing to pursue this degree
of integration.

% The following paragraph is what we want to claim, but is actually
% factually incorrect. A supercomputer center understands its
% cost to operate per unit time (e.g., cost to run per hour). This
% cost includes things like the cost of personnel, cost of renting
% the physical space for the supercomputer center, cost of electricity, etc.
% It turns out that the cost of electricity is a small percent of
% the overall cost of running the center. In the end, I think there
% is very little chance of getting supercomputer centers to
% willingly negotiate with electricity service providers because the cost of
% the lost opportunity is too high since the remaining costs are
% unchanged.
The responses to the questionnaire also suggest that some ESPs are requesting that their SC customers
develop capabilities for informing the provider of expected periods of
exceptional power consumption and for responding to requests from the
provider to consume less power for specified periods of time. Upon
initial consideration, this idea might seem to run counter to the
primary mission objective of most SCs of delivering
as many uninterrupted computational cycles as possible to their users.
In some extreme cases, SCs may not have a choice
in the matter as the size and energy requirements of supercomputers
increase; indeed, some ESPs may \textit{require} large
centers to develop a DR capability. However, a direct
business case may exist to encourage SCs to develop
this negotiation capability on their own. For example, if ESPs were to offer electricity at a significantly reduced rate
on the condition that the SC customer develop
DR capabilities, the long-term cost savings to the
center could make undertaking such a project worthwhile.

Perhaps one of the most straightforward ways that SCs can begin the process of developing a DR
capability is by enhancing existing system software used for managing
computing resources within the center. Indeed, the questionnaire
responses from Section~\ref{sec:questionnaire} as well as the literature
review presented in Section~\ref{sub:priorwork} both strongly support the
idea that the greatest opportunities for SCs to
develop integration capabilities are related to system software.
Specifically, and presented in approximate order of decreasing
interest and expected impact to the questionnaire respondents, system
software in this context consists of coarse-grained power management
(such as uniform processor power capping across the cluster), job scheduling, load migration, rescheduling backups, and fine-grained 
power management (such as dynamic, per-processor power capping).

Coarse-grained power capping may be one of the most straightforward
methods of power management. In the simplest form, this technique may
entail human intervention to adjust computing resources so they
operate at a reduced capacity or to entirely shut down some of the
computing capacity of a SC. By attenuating
resources, the SC manager can ensure that power
consumption stays below some defined level. This defined level may be
a pre-arranged power cap negotiated between the SC
and the ESP and maintained on an ongoing basis,
or, perhaps more likely, a power draw level that is requested by the
ESP to handle unanticipated loads somewhere else
in the ESP's system. 
Note that the savings
in power may not need to come entirely from attenuating computing
resources. Rather, reducing power consumption in computing resources
is likely to result in a corresponding reduction in thermal load
within the SC, which
may allow significant power savings in the cooling system as well.

The coarse-grained power capping technique described above assumes
that the SC environment has some amount of
instrumentation and metering that allows for the collection of power
telemetry data. This telemetry is necessary for the SC facility manager in order to understand how the power supplied
by the ESP is distributed to resources within the
center. Further, this telemetry is likely important to automated
solutions for power management, such as the job scheduling techniques
described below. In light of the fact that many system integrators
such as Cray and IBM are now delivering supercomputing systems that
include telemetry capabilities, the assumption that this information
is available seems acceptable. According to the responses to the
questionnaire presented in the previous Section, SC
facility managers perceive this accounting data as distinct from
per-user or per-job accounting data that is typically collected and
indicate that this data should be retained for electricity
provisioning planning purposes.

Techniques that involve job scheduling may offer more automated
approaches to power management. Due to the unique role that the job
scheduler and resource manager play within a SC,
these techniques may involve adjusting either the workflow of jobs
within the center or characteristics of the computational resources
within the center.

On one hand, the job scheduler has knowledge of and control over the
upcoming workflow within the SC simply by examining
and manipulating the job queue. One easily-accessible technique is
for a human operator to use capabilities such as advanced reservations
to reserve pre-arranged blocks of time in which jobs with high power
loads will run. These blocks of time could be negotiated with the
ESP on an ongoing basis or could be in response to
on-demand requests made by the ESP. Even more
automatic techniques are possible if the job scheduler is given enough
information about the workflow to make intelligent decisions about job
scheduling. For example, jobs may be submitted with various metadata
that enable the job scheduler to understand characteristics of each
job such as \textit{priority}, the relative importance of a job
compared to other jobs, and \textit{urgency}, the rate at which the
value of a job decreases as time elapses. These characteristics are
not only important to a job scheduler for ensuring efficient
utilization of a SC's resources under traditional
circumstances, but they are also a vital piece of successfully
implementing a DR capability for at least two reasons.
First, they provide a set of metrics by which the SC
can estimate the cost in terms of the ``lost opportunity'' of
responding to an ESP's request to run with
attenuated resources. Second, they allow the SC to
prioritize jobs in the queued workflow in order to understand how to
best utilize computational resources. This capability is important
under normal circumstances, but becomes even more essential in a
DR scenario.

At a lower level, schedulers and runtime systems can exercise 
fine-grained, dynamic DR capability. 
For example, the job scheduler knows which
nodes within a supercomputer are occupied with running jobs or are
expected to become occupied in the near future. To that end, the job
scheduler can use its control over the resource management process to
place idle nodes into a sleep state in which they draw significantly
reduced power. This strategy is especially effective in supercomputing
environments containing at least some resources that are used at
irregular intervals, allowing opportunities to utilize sleep states
effectively during periods when the resources are idle. 

In environments where all computing resources are heavily utilized,
fine-grained power scheduling will be directed by the runtime system.
For example, in the presence of load imbalance within a job, traditional
applications may rely on periodically moving data around the allocated
nodes to ensure all processors are performing a roughly equal amount
of work. This load-balancing process is both time- and energy-intensive.
By relocating power instead of data, processors with lighter loads can
surrender power and run slower, allowing more heavily-loaded processors
to use additional power to run faster. Combining both techniques 
should lead to improved execution time as well as more efficient
power utilization.

Even more interesting scenarios are possible in cases where the job
scheduler combines its knowledge of the upcoming queued workflow with
its knowledge and control over the computational resources within the
SC. These scenarios are most appropriate when the
supercomputing scenario contains a pervasively heterogeneous mix of
computational resources. For example, many contemporary SCs contain several different types of compute nodes with various
types of processors and accelerator cards. In some circumstances, the
job scheduler may be able to choose which resource to use for running
a given job among several candidate resources. The trade-off here is
not only in terms of the time necessary to complete the job (that is,
different resources could potentially complete the job in very
different amounts of time) but also in terms of the energy consumed in
completing the job (that is, different resources could potentially
consume very different amounts of energy in completing the job).
Further, other resources such as memory access patterns, disk access
patterns, and network use affect the energy signature of a job and may
be observed by the scheduler. By maintaining a database of
job-to-resource mappings that record the time and energy taken for
each job, the scheduler can, over time, improve its ability to decide
which jobs have the highest affinity to each type of resource. Using
this knowledge to optimize a SC's workflow in terms
of job throughput or energy consumption is admittedly complex, but the
potential rewards are likely to be compelling both to the day-to-day
operation of the center and to DR capabilities.

Opportunities may also exist for SCs to cooperate
with each other in scenarios in which computational loads are migrated
from one site to another where energy costs are less expensive. This
scenario is challenging for both technical and business reasons.
Technical challenges include issues such as user authentication and
authorization (i.e., a user may be authorized to use resources at one
site but not at another site) and data movement (i.e., it may be
infeasible to migrate large datasets from one site to another site).
To some extent, some of these technical challenges may be mitigated by
the use of advanced reservation capabilities in the scheduling systems
at each site, allowing resources to be simultaneously reserved while
large datasets are properly staged. Business challenges include the
notion that a SC currently has little incentive to
migrate jobs to another ``competing'' center. Indeed, the
questionnaire results reflect low interest in load migration
strategies. It seems likely that in order to be a feasible scenario,
the structure of payment and rewards to a SC to
cooperate with other centers would need to be structured differently
than they are currently.

In a very broad sense, DR techniques such as job
scheduling, power capping, and load migration can be considered to
be coarse-grained approaches because they involve considering ``big
picture'' views of the workload and computational resources in a
SC. According to the questionnaire results
presented in the previous Section, facilities managers view these
approaches as the most likely candidates for creating effective
DR capabilities.

Finally, this Section has focused heavily on the opportunities
available to SCs that come from developing
DR capabilities. This notion is primarily due to the
fact that the questionnaire presented in Section~\ref{sec:questionnaire}
was distributed to SCs in the United States, not to
ESPs. That said, opportunities do exist for
ESPs that develop DR
capabilities. At one level, the negotiation process itself requires
integration in terms of the communication and messaging protocols that
are necessary. To that end, opportunities exist for adapting and
extending existing standards currently used within the industry, thus
creating new use cases and capabilities for ESPs. At a higher level, ESPs will most
likely need to improve their ability to determine in near real time
the important places within the electrical grid where demands exceed
supply. Determining this is likely to be a complex optimization
problem. While this Section focuses on solving these problems to the
end of developing a DR strategy in conjunction with
SCs, these capabilities are likely applicable to a
broad range of customers.
