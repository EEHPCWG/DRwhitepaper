\documentclass[runningheads]{llncs}
%
\usepackage{makeidx}  % allows for indexgeneration
\usepackage[pdftex]{hyperref}
\usepackage{graphicx}
%
\begin{document}
%
\frontmatter          % for the preliminaries
%
\pagestyle{headings}  % switches on printing of running  heads
%
% test line number 2
%
\mainmatter              % start of the contributions
%
\title{
The Electrical Grid and Supercomputer Centers:  
An Investigative Analysis of Emerging Opportunities and Challenges
}

%
\titlerunning{The Electrical Grid and Supercomputing Centers}  % abbreviated title (for running head)
% also used for the TOC unless 
% \toctitle is used
%
%
\author{Fname Lname\inst{1} \and Fname Lname\inst{2}
Fname Lname \and Fname Lname \and Fname Lname \and Fname~M.~Lname \and
Fname Lname}
%
\authorrunning{Fname Lname et al.} % abbreviated author list (for running head)
%
%
\institute{Institute Name, City State Zip, Country,\\
\email{Name@email.adr},\\ WWW home page:
\texttt{http://web/page.html}
\and
Another Institute,
Institute Department, address1,\\
address2, Country}

\maketitle              % typeset the title of the contribution

\begin{abstract}
Some of the largest supercomputer centers in the United States are developing new relationships with their electricity service providers. 
These relationships, similar to other commercial and industrial facilities, are driven by mutual interest to save energy costs and improve electricity grid reliability.
Supercomputer centers are concerned about electricity price, quality, environmental impact and availability.
Electricity service providers are concerned about supercomputer center’s impact on the electrical grid reliability; for energy consumption, peak power and fluctuations in power.
Supercomputer center power demand can be greater than 20 megawatts (MW) or more - the theoretical peak power requirements are greater than 45MW - and re-occurring intra-hour variability can exceed 8MW.
As a consequence, there are some supercomputer centers whose electricity service providers are asking for hourly forecasts of power demand, a day in advance.
This paper explores today’s relationships, potential partnerships and possible integration be- tween supercomputer centers and their electricity service providers, and its value.
It evinces the challenges for this possible HPC-Grid integration, and then cites opportunities to overcome the barriers using feedback from a questionnaire of Top 100 List sized supercomputer centers in the United States.
\end{abstract}

%
\section{Introduction and Background}
Supercomputer centers with petascale systems for high-performance 
computing (HPC) are realizing the large impact they will be 
putting on their electricity providers with peak power demands of 20MW and instantaneous power fluctuations of 8MW. 
 
The Energy Efficient HPC Working Group 
(\href {http://eehpcwg.lbl.gov/}{EE HPC WG}) 
has been investigating opportunities for large supercomputer sites to more closely 
integrate with their electricity providers. This paper documents the 
results of this investigative activity. 

The EE HPC WG Team took as their starting point a model developed by
Lawrence Berkeley National Laboratory (LBNL's) Demand Response Research Center
that describes challenges and opportunities in which data centers and electricity
service providers may interact and how this integration can advance new market opportunities. \footnote{LBNL
Data Center Grid Integration Activities: http://drrc.lbl.gov/projects/dc} This integration model
describes programs that are used by some of the electricity service providers to encourage particular
responses by their customers and methods used to balance the electric grid supply and demand.
The findings also describe strategies that data centers might employ for utility programs to modify
their electricity and power requirements to lower costs and benefit from utility incentives. The EE HPC WG Team
adopted this model with slight tweaks to reflect the supercomputing environment focus 
(versus the data center as described by LBNL's Demand Response Research Center).

%\href{http://drrc.lbl.gov/publications/demand-response-and-open-automated-demand-response-opportunities-data-centers}
%{http://drrc.lbl.gov/publications}

This paper distinguishes the supercomputer center
as having unique characteristics that distinguish it from the data center. As opposed to data centers, 
supercomputer centers 
have very high system utilization and are not likely to use virtualization as a strategy.  
Also, supercomputer applications are generally not easily portable between 
geographic locations for a variety of reasons; security, data-locality, system tuning. 
Therefore, although included in the LBNL model for data centers, both virtualization and geographic load 
shifting were eliminated as
potential supercomputer center strategies.


Section 2 of this paper describes in greater detail the model for
integrating supercomputer centers and the electrical grid. Section 3 
is a review of prior work on HPC center strategies that might be
deployed for managing electricity and power. In order to further understand
today's relationships, potential partnerships and possible integration
between HPC centers, their electricity providers and the grid, a
questionnaire was deployed whose respondents were Top100 List class
supercomputer centers in the United States. Section 4 of this paper
describes the results of that questionnaire. The fifth section of the paper
describes opportunities, solutions and barriers. A sixth section describes
conclusions and next steps. Finally, the last section recognizes additional
authors.


\label{sec:intro}

\section{Supercomputer Centers and Electrical Grid Integration}
The EE HPC WG Team took as their starting point a
model developed by LBNL's Demand Response Research
Center that describes strategies that datacenters might em-
ploy for utility programs to manage their electricity and
power requirements to lower costs and benefit from utility
incentives. The EE HPCWG Team adopted this model with
slight tweaks to reflect the supercomputing environment
focus (versus the datacenter as described by LBNL`s Demand
Response Research Center).

For purposes of this paper, we define supercomputer centers
as distinct from a datacenters as having 
significantly higher system utilization and thus little or no 
virtualization.  Additionally, supercomputer
applications are distinguished by their lack of geographical
portability due to security concerns, data size and machine-specific
optimization.  We also note  that supercomputing centers tend to be more
energy-efficient than datacenters.  
In our survey, no SC exceeded a power usage effectiveness (PUE) of $1.53$,
while the average data center falls between $1.91$ and $2.9$ (with $1.0$ being
the ideal).~\cite{FIXME}

\subsection{Electricity Provider Programs\\
and Methods}
\label{sub:EPP}
An electricity provider seeks to provide efficient and reliable generation, transmission, and 
distribution of electricity. Methods and programs employed by the electricity providers and their consumers 
are key to managing and balancing the supply and demand of electricity. While the \textit{methods} 
describe how 
electricity providers manage supply, the \textit{programs} describe the activities that 
the electricity providers 
can offer to their consumers to balance demand with supply.

Although critical to the eletricity service providers, methods 
are generally not visible to the consumer of the electricity because they
operate within the generation  or transmission stations.
These methods are the major means by which supply and demand of electricity are managed.

Electricity provider programs encourage customer responses to target both energy efficiency and real-time
(day-ahead or day-of) management of demand for electricity. An example of an electricity provider program 
that encourages energy efficiency would be to provide home consumers a financial incentive for replacing 
single pane with double pane windows.  On the other hand, an example that illustrates programs that help 
with real-time demand management would be to provide a financial incentive for reducing load 
during high demand periods 
(such as hot summer afternoons when air conditioners are heavily utilized). 

The following is a list and brief definitions of key methods and programs.  

\subsubsection{Methods}
\begin{itemize}
\item Regulation (Up or Down): Methods used to maintain that portion of electricity generation reserves 
that are needed to balance generation and demand at all times.  Raising supply is up regulation and lowering 
supply is down regulation. There are many types of reserves 
(e.g., operating, ancillary services), distinguished by who manages them and what they are used for.

\item Transmission Congestion: Methods used to resolve congestion that occurs when there is not enough 
transmission capability to support all requests for transmission services. Transmission system operators 
must re-dispatch generation or, 
in the limit, deny some of these requests to prevent transmission lines from becoming overloaded.

\item Distribution Congestion:  Methods used to resolve congestion that occurs when the 
distribution control system 
is overloaded.  It generally results in deliveries that are held up or delayed.  

\item Frequency response:  Methods used to keep grid frequency constant and in-balance. 
Generators are typically used for frequency response, but any appliance that operates to a duty cycle 
(such as air conditioners and heat pumps) could be used to provide a 
constant and reliable grid balancing service by timing their duty cycles in response to system load.   

\item Grid Scale Storage:  Methods used to store electricity on a large scale. 
Pumped-storage hydroelectricity is the largest-capacity form of grid energy storage. 

\item Renewables:  Methods used to manage the variable uncertain generation nature of 
many renewable resources. 
\end{itemize}

\subsubsection{Programs}
\begin{itemize}
\item Energy Efficiency:  Programs used to reduce overall electricity consumption.

\item Peak Shedding:  Programs used to reduce load during peak times, 
where the reduced load is not used at a later time. 

\item Peak Shifting:  Programs where the load during peak times is moved to, typically, non-peak hours. 

\item Dynamic Pricing:  Time varying pricing programs used to increase, shed,
 or shift electricity consumption. 
The two types of pricing are peak and real-time.  Peak pricing is pre-scheduled; however, the consumer 
does not know if a certain day will be a peak or a non-peak day until day-ahead or day-of.  
Real-time pricing is not pre-scheduled; prices can be set day-ahead or day-of.
\end{itemize}

Although these methods and programs have historically not been relevant to supercomputer centers,
the following example illustrates their potential relevance.
The generation capacity requirements and response timescales vary across the country for electricity 
providers and operators. For example, the New England independent system operator (ISO-NE) uses a method 
of regulation and reserves that relies heavily on a day-ahead market program. This provides an opportunity 
for demand side resources---like supercomputer centers with renewable energy sources---to participate in the 
market supplying the ISO-NE with electricity.  It also makes the ISO-NE particularly sensitive to major 
fluctuations in electricity demand, which, as discussed further in the questionnaire section, is an emerging 
characteristic of the largest supercomputer centers.  
\footnote {http://drrc.lbl.gov/sites/drrc.lbl.gov/files/LBNL-5958E.pdf}

This paper assumes that the given grid is a constant. However, it is expected 
that future grid infrastructures will 
evolve with smart-grid capabilities. 

\section{Supercomputing Centers and \\ HPC-Grid Integration}

In November 2004, the Blue Gene/L system at Lawrence Livermore National Laboratory
became the fastest computer in the Top 500,~\cite{FIXME}, displacing the NEC Earth Simulator,
the previous champion.  This change marked the transition from supercomputing gains based
on ever-higher-performance components to systems comprised of far larger numbers of 
slow but energy-efficient components.  However, total system power consumption continued to rise,
and we are now poised to begin a second transition to ''power-limited computing''.  The new
model has been exemplified by the US Department of Energy issuing guidance that the first
DOE exascale machine should not exceed 20MW; effectively a $1000x$ performance improvement
with only a $3x$ increase in power.  

However, the problem is not as simple as provisioning 20MW.  Ultimately, SCs optimize for
performance per dollar, not performance per Watt, and flexibility in power consumption
can be expected to result in lower overall prices.  Use of green technologies such as
wind and solar may also lead to cheaper but less predictable sources of power.
To adapt to this new landscape, SCs may employ one or more strategies to control their 
electricity demand.

\begin{itemize}
\item \textbf{Node level.} Controlling power ultimately requires control of individual
components.  Historically, this control has been accomplished through Dynamic Voltage/Frequency
Scaling (DVFS), which allows the processor to use a lower voltage at the cost of a slower
clock frequency.  Newer technologies such as Intel's Running Average Power Limit leverage
DVFS to guarantee that at user-specified processor power bound will, on average, not be exceeded 
over the duration of a short time window.  DVFS can also be found on accelerator cards such
as nVidia's Kepler GPGPU.  Other efforts reduce DRAM power by batching reads and writes, thus
allowing the memory to spend more time in a lower-power state.  Several processor configuration
options have indirect but significant effects on power consumption.  For example, the choice
of the number of cores to use, whether or not to enable hyperthreading, and the use of 
"turbo" modes will change the power/performance curve.

\item \textbf{Job level.}  Each of the node-level controls requires a tradeoff between
power and  performance.  SCs resources are typically oversubscribed, so degrading performance
to save power and energy ultimately results in less science getting done.  However, at the 
job level, load imbalance provides opportunities to slow nodes that are off of the critical
path of execution without slowing the overall job execution time.  Traditionally, load 
rebalancing strategies have focused on moving bytes around the job allocation.  With 
power control, we can now rebalance power as well as work.

\item \textbf{System level.}  While most SCs use time and space partitioning (where a node
only runs a single job at a time), there are still shared resources that must be managed
across jobs.  Periodic checkpointing saves sufficient job state to a filesystem shared 
across jobs so that a job may be restarted from a recent point in case a fault occurs.
Because these checkpoints involve much more data motion than normal execution, power 
spikes can be observed at the node level (particularly DRAM), network, and filesystem.
These checkpoints may need to be coordinated across large jobs to prevent unnecessary
performance degredation.

\item \textbf{Scheduler level.}  Up through the system level, power control is evaluated
using the execution time of individual jobs.  The scheduler optimizes for overall throughput
rather than individual job performance.  At this point, scheduling is a two-dimensional 
problem:  jobs request a certain number of nodes for a certain duration.  As power-limited
computing becomes more common, schedulers will add power bounds to this mix:  a job will
be allowed nodes, time, and a certain number of watts (the responsibility for not exceeding
the job power bound rests with the system software, not the user or application).  The 
scheduler not only determines when jobs in the queue begin execution, but also what happens
when a job exits the system.  Depending on the priorities of already-running jobs and the
priorities of jobs in the queue, the best solution in terms of throughput may be to idle
the recently-freed nodes and redistribute the freed power to running jobs.

\item \textbf{Site level.}  At the level of the machine room (or multiple machine rooms),
decisions must be made as to how much power should be allocated for cooling versus computation,
which requires understanding how temperature interacts with performance.  A higher intake
air temperature uses less cooling power but results in higher static processor power and 
may limit opportunities for "turbo" mode in processors where it is available.  As cooling
power varies with outside air temperature, a single machine room temperature setpoint may 
not be the optimal solution in terms of overall performance.  

\end{itemize}


\label{sec:supercomputer}

\section{Prior Work}
We now describe prior work done in Power and Thermal Management, Job Scheduling, and Load Migration in the HPC and data center communities. Software and hardware techniques to save energy and power have been studied extensively. However, most of this work does not take into consideration power management strategies (especially for cooling systems and IT equipment) in response to a request from an electrical service provider \cite{Ghatikar2012a}.

\subsection{Power Management}
DVFS and power capping are two popular ways to manage node power. Prior work in the HPC domain looked at analytical models to understand energy consumption \cite{SpringerPPoPP2006,GeICPP2007,LiHPCA2006} and at trading execution time for lower power/energy \cite{CameronSC2005,HsuSC2005}. Several DVFS algorithms have also been proposed, such as CPUMiser \cite{GeICPP2007} and Jitter \cite{KappiahSC2005}. Varma et al, 2003 \cite{varma_control-theoretic_2003} demonstrated system-level DVFS techniques. They monitored CPU utilization at regular intervals and performed dynamic scaling based on their estimate of utilization for the next interval. Springer et al.~\cite{springer:06} analyzed HPC applications under an energy bound. Rountree et al. used linear programming to find near-optimal energy savings without degrading performance \cite{rountree:07} and implemented a runtime system based on this scheme \cite{rountree:09}. 

There also has been work in the real-time systems community to solve the DVFS scheduling problem using mixed integer linear programming on a single processor\cite{IshiharaISLPED1998,SaputraLCTES2002,SwaminathanRTSS2000,SwaminathanASPDAC2001}. Other real-time approaches looked at saving energy \cite{MoncusiRTSS2003,MochockiICCAD2002,MochockiRTAS2005,ZhuTPDS2003,ZhangDAC2002}. 

In addition, there has been active research in the domain of virtual machines. Von Laszewski et al. \cite{von_laszewski_power-aware_2009} presented an efficient scheduling algorithm to allocate virtual machines in a DVFS-enabled cluster by dynamically scaling the supplied voltages. Dhiman et al. designed vGreen \cite{dhiman_vgreen:_2009}, which is a system for energy efficient computing in 
virtualized environments. They linked online workload characteristics to dynamic VM scheduling decisions and achieved better performance, energy
efficiency and power balance in the system. Curtis-Maury et. al \cite{Curtis1,Curtis2,Curtis3} introduced Dynamic Concurrency Throttling, which is a technique to dynamically optimize for power and performance by varying the number of active threads in parallel codes. 

Chip power measurement and capping techniques were initially introduced with the Running Average Power Limit (RAPL) interface on Intel Sandy Bridge processors \cite{IntelSDM,David2010}. In the HPC domain, Rountree et al.~\cite{Rountree2012} proposed RAPL as an alternative to DVFS and analyzed application performance under hardware-enforced power bounds. They also established that variation in power directly translates to variation in application performance under a power bound. Patki et al. \cite{Patki1} used power capping techniques to demonstrate how hardware overprovisioning can improve HPC application performance under a global power bound significantly. Overprovisioning was also explored in the data center community \cite{femal:04}.

\subsection{Thermal Management}
Thermal and cooling metrics are becoming important in HPC resource management. Runtime cooling strategies are mostly job-placement-centric. These techniques either aim to place incoming computationally intensive jobs in a thermal-aware manner on servers with lower temperatures or attempt to migrate or load-balance jobs from high-temperature servers to servers with lower temperatures.

Kaushik et. al \cite{kaushik_t*:_2012} proposed \emph{T*}, a system that is aware of server thermal profiles and reliability as well as data semantics (computation job rates, job sizes, etc). This system saves cooling energy costs by using thermal-aware job placements without trading off performance.

Sarood et. al \cite{SaroodSC11} designed a runtime system that does temperature-aware load balancing in data centers using DVFS and task migration. They also discussed how hotspots could be avoided in data centers, and showed cooling costs can be reduced by up to 48\% with temperature-aware load balancing.

\subsection{Job Scheduling}
The problem of scheduling jobs has been extensively studied. Most resource managers implement the First Come First Serve (FCFS) policy
as a simple but fair strategy for scheduling jobs. However, FCFS suffers from low system utilization. A common optimization is backfilling
\cite{lifka_anl/ibm_1995,mualem_utilization_2001,feitelson_parallel_2004}. Backfilling improves system utilization by executing jobs with small resource requests out of order on idle nodes.

Fan et al. \cite{PowerAwareServer1} discussed power-aware job scheduling in the data center domain. 
They discussed a power monitoring system that could use power capping (based on a power estimation method such as RAPL or direct power sensing) and a power throttling mechanism. Such as system works well when is a set of jobs with loose service level guarantees or low priority that can be
forced to reduce consumption when the datacenter is approaching the power cap value. Etinski et al. \cite{Etinski1,Etinski2,Etinski3,Etinski4} explored scheduling under a power budget in supercomputing and analyzed bounded slowdown of jobs. In their series of papers, they introduced three policies. Their first policy is based looks at current system utilization and uses DVFS during job launch time to meet a power bound. Their second policy meets a bounded slowdown condition without exceeding a job-level power budget. Their third policy improves upon the former by analyzing job wait times and adding a reservation condition. 

There are many use cases in a grid computing environment that require QoS
guarantees in terms of guaranteed response time, including time-critical
tasks that must meet a deadline. Foster et. al \cite{foster_distributed_1999,foster_anatomy_2001} proposed \emph{advance reservations} to achieve time guarantees. Advance reservation is a guarantee for the availability of a certain amount of resources to users and applications at specific times in the future. The advance reservation feature requires scheduling systems to support reservation capabilities in addition to backfilling-based batch scheduling. Modern resource management systems such as Sun Grid Engine, PBS, OpenPBS, Torque, SLURM, Maui, and Moab support advance reservation capabilities.

\subsection{Load Migration}
Chiu et. al \cite{chiu_electric_2012} discussed a electrical grid balancing problem that was experienced in the Pacific Northwest. In order to match electricity supply and balance the electrical grid, they proposed low-cost geographic load migration. They also suggested that a symbiotic relationship between datacenters and electrical grid operators that leads to mutual cost benefits could work well.  Ganti et al. \cite{Ghatikar2012b} looked at two applied cases for distributed data centers. The results show that load migration is possible in both homogenous and heterogeneous systems. Their migration strategies were based on a manual process and can benefit from automation.

\subsection{Dynamic Pricing and Job Scheduling}
Aikema et. al \cite{aikema_electrical_2011} explored the potential for HPC centers to adapt to dynamic electrical prices, to variation in carbon intensity within an electrical grid, and to availability of local renewables. Their simulations demonstrated that 10- 50 \% of electricity costs could potentially be saved. They also concluded that adapting to the variation in the electrical grid carbon intensity was difficult, and that adapting to local renewables could result in significantly higher cost savings.

Power-aware resource management without degrading utilization has been proposed as a DR strategy to reduce electricity costs \cite{yang_integrating_2013,zhou_reducing_2013}. The novelty of the proposed job scheduling mechanism is its ability to take the variation in electricity price (dynamic pricing) into consideration as a means to make better decisions about job start times. Experiments on an IBM Blue Gene/P and a cluster system as well as a case study on Argonne's 48-rack IBM Blue Gene/Q system have demonstrated the effectiveness of this scheduling approach. Preliminary results show a 23\% reduction in the cost of electricity for HPC systems.

\label{sec:priorwork}

\section{Questionnaire} 
We used a questionnaire to understand the current experiences of interaction between
SCs and their ESPs. We restricted the analysis to sites in the United States
because the results of the survey and practices of demand response are highly
correlated and driven by energy policies in the country. 
\cite{torriti_demand_2010}.

Nineteen Top100 List sized sites in the United States were targeted for the
questionnaire. Eleven sites responded---Oak Ridge National Laboratory (ORNL), 
Lawrence Livermore National Laboratory (LLNL), 
Argonne National Laboratory (ANL), 
Los Alamos National Laboratory (LANL), 
Lawrence Berkeley National Laboratory (LBNL), 
Wright Patterson Air Force Base,
National Oceanic Atmospheric Administration (NOAA), 
National Center for Supercomputing Applications (NSCA), 
San Diego Supercomputing Center (SDSC), 
Purdue University and Intel Corporation. The questionnaire was
sent to a sample that was not randomly selected. It was sent to those sites
where it was relatively easy to identify an individual based on membership
within the EE HPC WG. The sample is more representative of Top50 sized sites
(One Top50 sized site was not in the sample and 60{\%} (9/15) of the sample
responded). Only 4 additional sites were sampled from the Top51-Top100 List
and, of those, 2 responded (Intel and NOAA).
%Patki: New Graph instead of Table 1
\begin{figure}[htbp]
\begin{center}
\includegraphics[scale=0.45]{NewGraphs/Table1-graph.pdf}
\caption{Site Load and Variability}
\label{figGraph1}
\end{center}
\end{figure}

The total power load as well as the intra-hour fluctuation of these sites
varied significantly (Figure \ref{figGraph1}). Total power load includes all computing systems plus ancillary systems such as power delivery and cooling components.
There were four sites with total power load greater
than 10 MW, two sites with \textasciitilde5 MW total power load and five
sites with less than 2 MW  of total power load. For those with total power load greater than 10 MW, the
intra-hour fluctuation (maximum variability) varied from less than 3 MW to 8 MW. One of
\textasciitilde 5 MW sites said that they experienced 4 MW variability. We chose less than 3 MW intra-hour variability as the bottom of the scale because we assumed that the
ESPs would not be affected by 3 MW (or less) 
fluctuations. The rest of the sites all reported less than 3 MW intra-hour fluctuation. Most of the intra-hour variability was due to preventative maintenance. 
%TP: THis is repeated. (again, the power variation includes both computing and ancillary systems).

For every respondent, the theoretical peak energy or maximum load is approximately twice the total energy, which is indicative of expected future growth in power and energy requirements for SCs. Some of the design parameters that may affect theoretical peak limits are the customer switchgear, transformer 
and chiller water capacities. In some cases, there are also limits based on regional ESP capacity constraints.

We asked if the SCs had talked to their ESPs about programs and methods used to balance the grid supply and demand of electricity (see Table 1). About half of them have had some discussion, but it
has mostly been limited to programs (e.g., peak shed, dynamic pricing) 
and not methods (e.g., regulation, frequency response, congestion).

\begin{table}[htbp]
\caption{Discussions with ESPs}
\begin{tabular}{|p{190pt}|l|}
\hline
\textbf{Discussions with ESPs}&
{\%} Yes \\
\hline
\textbf{Demand-side programs}&
~ \\
\hline
Shedding load during peak demand&
54 \\
\hline
Responding to pricing incentive programs&
45 \\
\hline
Shifting load during peak demand&
36 \\
\hline
\textbf{Supply-side programs}&
~ \\
\hline
Enabling use of renewables&
36 \\
\hline
Congestion, Regulation, Frequency Response&
18 \\
\hline
Contributing to electrical grid storage&
10 \\
\hline
\end{tabular}
\label{tab2}
\end{table}

Approximately half of the respondents are not currently interested in shedding
load during peak demand. LANL reports that the ``technical feasibility" and ``business case has yet to be developed."
There is slightly more interest in shifting than shedding load. SDSC reports that
 ``Automatic load shedding is being explored/deployed today'' for the entire campus, not just the supercomputing center.

\begin{table*}[htbp]
\centering
\caption{HPC Strategies Responding to Electricity Provider Requests}
\begin{tabular}{|p{2.5in}|p{0.75in}|p{0.75in}|p{0.75in}|} \hline

\textbf{HPC strategies for responding to} &
\textbf{\%} &
\textbf{\%} &
\textbf{\%} \\

\textbf{Electricity Provider requests} &
\textbf{Interested} &
\textbf{High} &
\textbf{Medium} \\

\textbf{(listed from highest to lowest interest + impact)} &
 &
\textbf{Impact} & 
\textbf{Impact} \\
\hline

Coarse grained power management &
64 &
46 &
27 \\
\hline

Facility shutdown&
36 &
64 &
10 \\
\hline

Job scheduling&
36 &
27 &
18 \\
\hline

Load migration &
10 &
36 &
18 \\
\hline

Re-scheduling back-ups &
45 &
0 &
10 \\
\hline

Fine-grained power management &
27 &
0 &
36 \\
\hline

Temperature control beyond ASHRAE limits &
27 &
0 &
18 \\
\hline

Turn off lighting &
18 &
0 &
0 \\
\hline

Use back-up resources (e.g., generators) &
0 &
10 &
27 \\
\hline

\end{tabular}
\label{tab3}
\end{table*}
Responding to pricing incentive programs is also not considered currently interesting to approximately half of the respondents, although the reasons for this low interest may be organizational. Several
open-ended comments revealed that pricing is fixed and/or done by another
organization at the site level and outside of their immediate control.

Only twenty percent of the respondents have had discussions with their
ESPs about congestion, regulation and frequency
response. LANL is one of two who have had discussions and who commented that
they are ``learning about the process'' and that it is ``outside of [their] visibility or control''.

There were many more respondents who have had discussions with their
ESPs about enabling the use of renewables; 36{\%}
have already had discussions and more than half are interested in further
and/or future discussions. SDSC already has a site-wide program; ``the
campus has a large fuel cell (2.5$+$ MW) and works with the utility with
renewables.'' Other responses suggest that the interest is at the site level
and not unique to the SC.

An open-ended question was posed as to whether or not there was information
either requested of the SCs by their ESPs or,
conversely, requested of the ESPs by the SCs. In both cases, well
over 75{\%} of the respondents answered no. LLNL and LANL were the
exceptions. LLNL is ``responding to requests for additional data on an hourly, 
weekly and monthly basis." They are also working to develop an automated capability to share 
data with their ESPs, which would provide automated additional 
detailed forecasting and ultimately real time data."
LANL has also been requested to provide average ``power projections, hour by hour,
for at least a day in advance.'' Additionally, LANL has asked their ESP for
more information on ``sensitivity of power distribution grid to rapid
transients (random daily step changes of 10 MW up or down within a single AC
cycle).''

Given the low levels of current engagement between the ESPs and the SCs, it is not surprising that none of
the SCs are currently using any power management
strategies to respond to grid requests by their ESPs. SDSC's \textit{supercomputer center} is not an exception, but they did respond that their
entire ``campus is leveraging parallel electrical distribution to trigger
diesel generators and other back-up resources to respond to grid and
non-grid requests.''

It was suggested by ORNL that some of the power management strategies 
are of questionable business value even for energy efficiency, let alone grid integration.
For example, ORNL comments that ``these assets have very clear depreciation schedules, and the modest cost 
savings in terms of electricity consumption due to some of these methods may not (or frequently will not) 
outweigh the capital investment cost in the computer. That is, if a site spent \$100M for a computer that will 
remain in production for 60 months, then the 
apparent benefit of power capping, etc can easily be outweighed by lost productivity of the consumable resource.

Similarly, another comment by ORNL suggested that the rapid deployment of hardware features, like P-states,  may outpace the need for strategies like power aware job scheduling.

We tried to evaluate if power management strategies will be considered
relevant and effective for grid integration at some point in the future. Two
questions were asked: is there interest in using the strategies and what
impact did they think that the strategies would have? When combining
interest and impact, the results showed that power capping, shutdown, and
job scheduling were both potentially interesting and of high impact (see Table 2). 

Load migration, back-up
scheduling, fine-grained power management and thermal management were of medium
interest and impact. Lighting control and back-up resources were of low
interest and impact. 
%Patki: Fig 2
\begin{figure}
\begin{center}
\includegraphics[scale=0.45]{NewGraphs/PUE-Graph.pdf}
\caption{Site Power Usage Effectiveness}
\label{figPUE}
\end{center}
\end{figure}

Temperature control and lighting management are utilized as strategies, but considered medium to low interest and impact
for responding to requests from electricity service providers. 
The infrastructure energy efficiency of the responding supercomputer sites is high, as reflected in their reported
Power Usage Effectiveness (PUE) (Figure \ref{figPUE}). Two sites reported a PUE below 1.25, the majority were between 
1.25 and 1.5 and the highest was 1.53. Approximately half of the respondents said that they used 
temperature control and lighting management
as strategies, but not for grid requests. Temperature control and lighting management are well documented and understood
strategies for improving energy efficiency, so it is not surprising that sites with PUEs below 1.5 are using them.

NOAA comments that their ``lights automatically shut off 24x7 
when there is no motion in the data center." There is a value in lighting control for 
energy efficiency purposes, as demonstrated by its having been fully implemented. NOAA also comments that
the impact of further lighting control "is so small compared to the HPC demand load that" they would ``be surprised 
if the utility is interested." 

LLNL reports that they ``took 3 years to raise the temperature in their center by 18 degrees F. 
It was done in conjunction with a failure rate analysis of the systems as well as a measurement of the 
electrical savings prior to moving to the next set point”. LLNL is currently operating in the ASHRAE 
recommended range, but expresses concerns with increasing temperature as a grid-integration response. The 
concerns include hardware failures, 
tape storage read/write errors and compromising dew point requirements where liquid and air-cooling are co-located.

Distinguishing interest from impact sheds further insight; some strategies are 
considered high impact, but not interesting enough to consider deployment. Facility
shutdown is rated as having a high impact, but only
considered interesting by 36\% of the respondents. NOAA commented that, ``We've had too many instability and equipment failures to utilize this as a strategy." This divide is even more
apparent with load migration. It is rated as having a high impact by 36\% of the
respondents, but only interesting to 10\%. 
\label{sec:questionnaire}

\section{Opportunities/Solutions and Barriers} 
The responses to the questionnaire presented in
Section~\ref{sec:questionaire} represent a variety of desires and
experience regarding interactions between supercomputing centers and
energy service providers.  For example, the responses from the two centers
with the largest power draws, Lawrence Livermore National Laboratory
(LLNL) and Oak Ridge National Laboratory (ORNL), diverge in several
areas.  This divergence is perhaps primarily due to characteristics of
their respective energy service providers.  In contrast, San Diego
Supercomputer Center (SDSC) stands out as a leader in integrating
with their energy service provider on a site-wide level.  To that end, the
responses from SDSC may exemplify some of the opportunities available
to other supercomputing centers that are willing to pursue this degree
of integration.

% The following paragraph is what we want to claim, but is actually
% factually incorrect.  A supercomputing center understands its
% cost to operate per unit time (e.g., cost to run per hour).  This
% cost includes things like the cost of personnel, cost of renting
% the physical space for the datacenter, cost of electricity, etc.
% It turns out that the cost of electricity is a small percent of
% the overall cost of running the center.  In the end, I think there
% is very little chance of getting supercomputing centers to
% willingly negotiate with energy service providers because the cost of
% the lost opportunity is too high since the remaining costs are
% unchanged.
The responses to the questionaire also suggest that some energy service
providers are requesting that their supercomputing center customers
develop capabilities for informing the provider of expected periods of
exceptional power consumption and for responding to requests from the
provider to consume less power for specified periods of time.  Upon
initial consideration, this idea might seem to run counter to the
primary mission objective of most supercomputing centers of delivering
as many uninterrupted computational cycles as possible to their users.
In some extreme cases, supercomputing centers may not have a choice
in the matter as the size and energy requirements of supercomputers
increase; indeed, some energy service providers may \textit{require} large
centers to develop a demand-response capability.  However, a direct
business case may exist to encourage supercomputing centers to develop
this negotiation capability on their own.  For example, if energy service
providers were to offer electricity at a significantly reduced rate
on the condition that the supercomputing center customer develop
demand-response capabilities, the long-term cost savings to the
center could make undertaking such a project worthwhile.

Perhaps one of the most straightforward ways that supercomputing
centers can begin the process of developing a demand-response
capability is by enhancing existing system software used for managing
computing resources within the center.  Indeed, the questionaire
responses from Section~\{sec:questionaire} as well as the literature
review presented in Section~\{sec:priorwork} both strongly support the
idea that the greatest opportunities for supercomputing centers to
develop integration capabilities are related to system software.
Specifically, and presented in approximate order of decreasing
interest and expected impact to the questionaire respondents, system
software in this context consists of coarse-grained power management
in the form of power capping, job scheduling, load migration,
rescheduling backups, and fine-grained power management.  Of these,
job scheduling may be a practical starting point simply because of the
unique role that the job scheduler and resource manager play within a
datacenter.

% I (koenig) would be able to self-cite out of my own publications
% concepts such as ``priority'' and ``urgency'' described in the
% following paragraph.  If desired, we would need to add these
% papers into the bibliography and/or related works section.
On one hand, the job scheduler has knowledge of and control over the
upcoming workflow within the supercomputing center simply by examining
and manipulating the job queue.  For example, jobs may be submitted
with various metadata that enable the job scheduler to understand
characteristics of each job such as \textit{priority}, the relative
importance of a job compared to other jobs, and \textit{urgency}, the
rate at which the value of a job decreases as time elapses.  These
characteristics are not only important to a job scheduler for ensuring
efficient utilization of a supercomputing center's resources under
traditional circumstances, but they are also a vital piece of
successfully implementing a demand-response capability for at least
two reasons.  First, they provide a set of metrics by which the
supercomputing center can estimate the cost in terms of ``lost
opportunity'' of responding to an energy service provider's request to
run with attenuated resources.  Second, they allow the supercomputing
center to prioritize jobs in the queued workflow in order to understand
how to best utilize computational resources.  This capability is
important under normal circumstances, but becomes even more essential
in a demand-response scenario.

On the other hand, the job scheduler has knowledge of and control over
the computational resources within the supercomputing center.



Taken together, the knowledge the scheduler has, described in the
previous two paragraphs, is important for power capping.



% need to expose knobs into the system software so that HPC facility
% managers can easily adjust the objectives that the system software
% is using to make decisions because the overall number of ways of
% scheduling a workflow makes the problem too hard to readily solve by
% hand
%
% system software: create the control points for measuring power consumption
% and affecting changes in the way energy is used throughout a datacenter
% (most likely, based on workflow); once these control points exist within
% an infrastructure, it is then possible to build site-specific policies on
% top of the control infrastructure
%
% also in this thought process, mention metering and measuring so we can
% understand what the energy service provider has delivered to us and where it
% is going

Load migration requires cooperation between HPC centers (and also may
require overcoming difficult challenges like migrating large datasets,
ensuring security requirements are met and porting tuned code).  By
using \textit{advance reservation capabilities} of schedulers (within
local resource managers) of HPC centers, we facilitate \textit{the
execution of load migration strategy} between HPC centers (e.g., in
terms of automation); as a result we increase the interest level and
to some extent the impact level of load migration strategies.


just lowering the power consumption of a batch job (i.e., by using
fine grained power management techniques) does not ensure that the
overall energy consumption is reduced; need some kind of knowledge
about the workflow in the organization to make these kinds of
determinations





if there is some kind of process that takes place between utility
providers and HPC centers, s likely that the utility providers will
need to improve their capabilities to be able to participate in this
process (e.g., probably need to solve some kind of weighted
optimization problem in near real time in order to know where the most
important places are to ensure uninterrupted service); there is an
opportunity to the utility provider in this, however, in that these
advancements in their technology for monitoring and adjusting their
infrastructure might be leveraged toward other ends that are not
related to HPC centers specifically





%%%%%%%%%%%%%%%%%%%%%%%%%%%%%%%%%%%%%%%%%%%%%%%%%%%%%%%%%%%%%%%%%%%%%


%% %negotiation/interaction 
%% The biggest opportunity in the integration of the electrical grid and
%% supercomputer centers is to start a process of negotiation/interaction
%% between energy service providers and HPC centers; the survey data indicates
%% that this is being asked for by providers, at least in some small way
%% already.

%% SDSC stands out as a leader in integrating with their energy service
%% provider.  They have implemented a site-wide integration.  A probably
%% opportunity for other supercomputer centers would be to pursue a
%% similar site-wide integration.

%% The two largest power draw sites (LLNL and ORNL) seem to have
%% divergent experiences.  Not clear why, but probably has more to do
%% with their providers than their sites.  LLNL's experience may be seen
%% with other sites -- LANL seems to be moving in that direction.  This
%% may portend a \textit{required} opportunity for some sites.

%% There isn't a clear business case on the part of the supercomputing
%% center for pursuing integration.  This probably needs to be further
%% understood and developed.


%% %system software
%% According to the survey data and literature review, the greatest
%% opportunities for HPC centers to develop integration capabilities are
%% related to system software.  System software in these areas in the
%% order of their interest and impact consist of coarse grained power
%% management (power capping), job scheduling, load migration,
%% rescheduling back-ups and fine grained power management.

%% Sites are developing experience with energy efficiency that can
%% transfer to power management for utility integration.

%% Job scheduling is seen as having the greatest interest and impact
%% because, from a practical perspective, coarse-grained power management
%% is dependent upon job scheduling.

%% If integration starts happening between energy service providers and HPC
%% centers, the system software (i.e., job scheduler) is a key component
%% in order to ensure that this happens as efficiently as possible in
%% order to keep high utilization / business utility going at the HPC
%% center.  (Consider here things such as fluctuations in HPC use; e.g.,
%% things like large-scale acceptance / Top500 style runs).  Load
%% migration requires cooperation between HPC centers (and also may
%% require overcoming difficult challenges like migrating large datasets,
%% ensuring security requirements are met and porting tuned code).  By
%% using \textit{advance reservation capabilities} of schedulers (within
%% local resource managers) of HPC centers, we facilitate \textit{the
%% execution of load migration strategy} between HPC centers (e.g., in
%% terms of automation); as a result we increase the interest level and
%% to some extent the impact level of load migration strategies.


%infrastructure software
In addition to system software, infrastructure software such as grid
computing may have a place in opportunities.  In the US, supercomputer
centers are connected via grid computing infrastructures such as
TeraGrid, Open Science Grid.  Grid computing's protocols, interfaces,
and standards can facilitate the execution of DR strategies, as a
result grid computing may increase the interest level and/or the
impact level of DR strategies.

If there is some kind of automated ``negotiation'' process that takes
place between energy service providers and HPC centers, it's likely that the
energy service providers will need to improve their capabilities to be able
to participate in this process (e.g., probably need to solve some kind
of weighted optimization problem in near real time in order to know
where the most important places are to ensure uninterrupted service);
there is an opportunity to the energy service provider in this, however, in
that these advancements in their technology for monitoring and
adjusting their infrastructure might be leveraged toward other ends
that are not related to HPC centers specifically.


%grid methods like dynamic pricing
The survey data indicates that grid programs are being negotiated
between energy service providers and HPC centers more than that of grid
methods.  Due to the lack of a clear business case there is low
interest in shedding and shifting load during peak
demands.  Nonetheless, according to the survey data in HPC landscape
load shifting is more attractive than shedding load.  In addition,
there are high interests and opportunities in the use of renewables at
the site level, according to the survey data and literature
review.  According to literature review, there are great opportunities
in terms of reducing electricity costs in exploiting dynamic pricing
as a grid integration program.

In contrary, there is lack of knowledge on the use and integration
aspects of congestion, regulation, and frequency response methods.

%accouting
There are needs and opportunities for the supercomputer sites in
having accounting system.  According to the survey data, there are
information request on electricity usage by electricity providers from
the supercomputer sites for getting detailed forecasting and real time
data.  Accounting data can be used to forecast and model future energy
usage of an HPC center.  So this can be communicated and be integrated
with electricity grid.  Accounting data can be classified in terms of
HPC center components, cooling, systems, lighting, etc.  If/how
electricity grid providers can use energy and usage accounting data to
plan electricity provisioning of an HPC center? %user-specific
accounting data versus workload-specific accounting data.


%electricity-price markets
%interoperability
The HPC/electrical grid integration at the communication and message
level need to be standardized and be interoperable.  Interfaces,
communication infrastructure, data, information exchange, and
agreement should be based on standards.

\label{sec:opportunities}

\section{Conclusions and Next Steps}
This paper explores the possibility of a new relationship between electricity service providers
and supercomputer centers with increased communication and engagement from both parties.

Because supercomputer centers have an increasingly large and fluctuating
power demand, they challenge their providers to supply a reliable
source of electricity.
Electricity service providers are interested in partnering with customers,
like supercomputer centers, to create a more dynamic and resilient grid
by obtaining predictable demand forecasts and engaging in programs like 
demand response.

We focused our attention on the largest supercomputer centers in the United States. The two supercomputer 
centers with the largest electricity demand, ORNL and LLNL, have had very different experiences.  
ORNL's experience is that its electricity demand and fluctuations are not significant factors for their 
electricity service provider.  
LLNL's experience is opposite to that of ORNL.  Because of large swings in 
power usage, the LLNL supercomputer center was approached by their electricity service provider with a 
request for daily predictable demand forecasts. That request began an ongoing relationship.  

The LANL supercomputer center's experience is similar to that of LLNL. SDSC has an even tighter relationship 
with their electricity service provider, but this relationship involves the entire campus and not just the 
supercomputer center.  

As previous research with datacenters has shown \cite{LBNL-6560E}, supercomputer centers can serve as   
resources to the grid. To enable this, automation technologies and data communication standards, 
which can link the supercomputer centers with the electric grid and on-site power management strategies 
for grid services will play a key role to ease adoption and lower the participation costs.  Power capping, 
shutdown, and job scheduling are identified as the most interesting management strategies with the highest 
leverage for responding to requests from electricity service providers.  

Nonetheless, the business case for the grid integration of supercomputer centers remains to be demonstrated.  
Supercomputer centers have concerns that deploying these strategies might have an adverse impact on 
their primary mission. One of the key enablers for supercomputing centers to participate in 
electricity markets (e.g., demand response, electricity prices) is having markets that value their 
participation. In other areas like commercial buildings and select industrial facilities, benefits to 
both electricity service providers and customers are well documented. However, as the electrical grid 
and new dynamic loads such as supercomputer centers evolve, the markets need mechanisms to identify 
and provide value of participation (e.g., cost, energy, carbon).

We are planning to pursue several areas in our future work.

We are planning a similar survey for Europe to explore if there is a more compelling business case 
in other geographies.
We expect the business value of such grid integration to be enhanced where the price of 
electricity is expensive, varies dynamically, or where there is strong reliance on expensive back-up 
generation (e.g., India).   

We plan on following-up with the ESPs that support these US-based supercomputer 
centers. We note that this work's focus was from the perspective of the supercomputer center, and 
we are interested in hearing from the ESPs about what makes a customer more or less interesting or 
challenging with respect to grid integration.

With increasing variable renewable generation and price-based DR programs, the intra-hour fluctuations 
and demand forecasting are becoming increasingly important.
Electrical grid programs may react in different ways to the timescale of a supercomputer center's load response.
What are the trends 
in inter-hour fluctuation patterns?  Is this a new behavior, an interim one, or one that is likely to get worse?


\label{sec:conclusion}

\newpage


%
% ---- Bibliography ----
%
\bibliographystyle{abbrv}
\bibliography{../bib/whitepaper,../bib/patki}  % eehpcmethod.bib is the name of the Bibliography in this case
%

\section{Appendices}
\input{appendices}
\label{sec:appendices}

\end{document}
