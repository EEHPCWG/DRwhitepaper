%This section is a work in progress
%
\begin{enumerate}
\item Potential HPC-specific value proposition for active DR engagement
\item Based on Grid Integration solutions -- local and system-wide impacts
\item Next steps -- specific directions or target areas to focus
\end{enumerate}

In past, electricity providers have viewed the hourly, daily, and seasonal
fluctuations of demand as facts of life. These fluctuations required
additional generating capacity, particularly peaking plants that were needed
only a few hours per year. 

However, with increasing adoption of Smart Grid, information technology, communications, and a more dynamic and reslient grid, 
the value proposition of replacing the expensive and fossil-based peaking plants with more predictable demand with forecasting and demand response has the potential for new value proposition to the HPC data centers. 
The increase in renewable generation and its intermittency offers opportunities for large and flexible loads that can adapt to changing electricity generation and sources.

DR adoption: 
Automation: 
%
As the previous research as shown with data centers, HPC can also be considered as a resource to the grid and provide different services.
To enable this, automation technologies, which can link the HPC data centers with the electric grid, 
and on-site power management strategies for different timescales for grid services will play a key role in ease of adoption and lowering the participation costs.

Electriicty markets: 
One of the key enabler for HPC data centers to participate in electriicty markets (e.g., demand response, electricity prices)
is the markets that value their participation. 
In other areas of commercial buildings and select industrial facilities, benefits to both electricity service providers and customers are well documented.
However, as the electric grid and new dynamic loads such as HPCs evlove, the markets need mechanisms to identify and provide value of participation (e.g., cost, energy, carbon). 
(LIST SOME KEY FINDINGS FROM QUESTIONNAIRE)

value proposition, (INCLUDED IN THE ELECTRICITY MARKETS ABOVE)

the measurement and verification models, (LISTED BELOW)

patents, (NOT SURE WHAT THIS IS AND WHY THIS IS RELEVANT)

intellectual property (NOT SURE WHAT THIS IS AND WHY THIS IS RELEVANT)

lighting, and heating, ventilation, and air conditioning (HVAC)
It is clear from the HPC operation that the largest opportunity for load management exists in the IT equipment.
However, the previous work suggests that the opportunity in HVAC loads may also be an immediate opportunity. 
The opportunity depends on the ratio of cooling loads to the IT equipment load (e.g., PUE).
Considering that the lighting loads are a small percentage of overall HPC load, their participation alongside other strategies (IT equipment and HVAC) may prove beneficial. 
The advancement in technologies and vendor recognition to provide more dynamic power management capabilities in the IT equipment offers larger opportunity in consolidated IT and cooling load reductions.

electricity-price markets (INCLUDED IN THE ELECTRICITY MARKETS ABOVE)

interoperability

The grid integration need to be standardized and provide interoperable
interfaces to be interoperable. Interfaces, communication infrastructure,
data, information exchange, agreement should be based on standards.
Communication with grid providers need to be standardized. grid
request/response messages. Requests include DR event, price, renewable
generation


How is architected an accounting system (energy and utilization) of an HPC
center? based on sensor systems like in [Hay09] . Sensor systems for an HPC
center to report real time power consumption of various components such as
cooling, compute systems, storage, networks, racks, etc.

[Hay09] S. Hay and A. Rice, ``The case for apportionment,'' in Proceedings
of the First ACM Workshop on Embedded Sensing Systems for Energy-Efficiency
in Buildings, New York, NY, USA, 2009, pp. 13--18.

Apportioning the total energy consumption of a building or organization to
individual users may provide incentives to make reductions. We explore how
sensor systems installed in many buildings today can be used to apportion
energy consumption between users. We investigate the differences between a
number of possible policies to evaluate the case for apportionment based on
energy and usage data collected over the course of a year. We also study the
additional possibilities offered by more fine-grained data with reference to
case studies for specific shared resources, and discuss the potential and
challenges for future sensor systems in this area.

\begin{itemize}
\item If accounting data can be used to forecast and model future energy usage of an HPC center? so this can be communicated and be in
integrated with electricity grid.
\item If/how electricity grid providers can use energy and usage accounting data to plan electricity provisioning of an HPC center?
\item user-specific accounting data versus workload-specific accounting data.
\item accounting data in terms of HPC center components, cooling, systems, lighting, etc.
\end{itemize}
These are excellent questions. What you've outlined below is a set of value
of real-time data (the term "accounting" confused me earlier) of energy and
utilization for HPC systems. Some of these values are for EE and the rest is
how the electric grid service providers can benefit from it. For example,
telemetry data for wholesale DR markets and M{\&}V.

M{\&}V or Measurement and Verification refers to quantification of load shed
that a particular load is participating in. Typically, there are many
baseline methodologies that the utilities and ISOs use to calculate the
amount of DR a particular load/facility is providing through real-time and
day-ahead metered data. The metering and telemetry to provide the M{\&}V is
key in determining if a particular resource can participate in a DR market 
and validate its performance for settlement (economics).

