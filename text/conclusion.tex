This paper seeks whether or not a new relationship is emerging between electricity service providers 
and supercomputer centers with the impetus for more communication and engagement coming from both parties.  
Supercomputer centers have increasingly large electricity demand and fluctuations that can be a challenge 
for their providers and the reliable supply of electricity.  On the other hand, electricity service providers 
are interested in partnering with customers, like supercomputer centers, to obtain predictable demand 
forecasts and engage in programs like demand response in order to create a more dynamic and resilient grid.

We focused our attention on the largest supercomputing centers in the United States. The two supercomputing 
centers with the largest electricity demand, ORNL and LLNL, have very different experiences.  ORNL's 
experience is that their electricity demand and fluctuations are not significant factors for their 
electricity service provider.  LLNL's experience is opposite of ORNL.  Because of the large swings in 
power usage, the LLNL supercomputing center was approached by their electricity service provider with a 
request for daily predictable demand forecasts.   That initiated the development of an on-going relationship.  
LANL's supercomputing center's experience is similar to that of LLNL. SDSC has an even tighter relationship 
with their electricity service provider, but this relationship involves the entire campus and not just the 
supercomputer center.  

As the previous research with data centers has shown, supercomputer centers can also be considered 
resourceful to the grid. To enable this, automation technologies and data communication standards, 
which can link the supercomputer centers with the electric grid, and on-site power management strategies 
for grid services will play a key role to ease adoption and lower the participation costs.  Power capping, 
shutdown and job scheduling are identified as the most interesting management strategies with highest 
leverage for responding to requests from electricity service providers.  

The business case for grid integration of supercomputer centers though, has yet to be demonstrated.  
There are concerns about the impact that deploying these strategies might have on the primary mission 
of the supercomputer centers.  One of the key enablers for supercomputing centers to participate in 
electricity markets (e.g., demand response, electricity prices) is having markets that value their 
participation. In other areas like commercial buildings and select industrial facilities, benefits to 
both electricity service providers and customers are well documented. However, as the electric grid 
and new dynamic loads such as supercomputer centers evolve, the markets need mechanisms to identify 
and provide value of participation (e.g., cost, energy, carbon).

There are several areas suggested for future work that we are planning to pursue.

The business value of such grid integration is likely enhanced further where the price of 
electricity is expensive, varies dynamically or where there is strong reliance on expensive back-up 
generation (e.g., India).  We plan on conducting a similar survey in Europe, where electricity is more 
expensive, to explore if there is a more compelling business case in other geographies.    

This work's focus was from the perspective of the supercomputer center.  It might yield different 
results if we were to hear from the ESPs about what makes a customer more or less interesting or 
challenging with respect to grid integration.  How do they view supercomputer centers relative to their 
spectrum of customers?  We plan on follow-up with the ESPs that support these US-based supercomputer 
centers to explore this question further.

With increasing variable renewable generation and price-based DR programs, the intra-hour fluctuations 
and demand forecasting is becoming increasingly important. Understanding the timescales of supercomputer 
center's load response will have different value to the electric grid programs. What are the trends 
in inter-hour fluctuation patterns?  Is this a new behavior, an interim one, or one that is likely to get worse?  



