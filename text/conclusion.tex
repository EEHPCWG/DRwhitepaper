
\begin{enumerate}
\item Potential HPC-specific value proposition for active DR engagement
\item Based on Grid Integration solutions -- local and system-wide impacts
\item Next steps -- specific directions or target areas to focus
\end{enumerate}

Electricity providers have viewed the hourly, daily, and seasonal
fluctuations of demand as facts of life. These fluctuations required
additional generating capacity, particularly peaking plants that were needed
only a few hours per year.

DR adoption: automation

automation technologies

DR market:

value proposition,

the measurement and verification models,

patents,

intellectual property

lighting, and heating, ventilation, and air conditioning (HVAC)

electricity-price markets

interoperability

The grid integration need to be standardized and provide interoperable
interfaces to be interoperable. Interfaces, communication infrastructure,
data, information exchange, agreement should be based on standards.
Communication with grid providers need to be standardized. grid
request/response messages. Requests include DR event, price, renewable
generation


How is architected an accounting system (energy and utilization) of an HPC
center? based on sensor systems like in [Hay09] . Sensor systems for an HPC
center to report real time power consumption of various components such as
cooling, compute systems, storage, networks, racks, etc.

[Hay09] S. Hay and A. Rice, ``The case for apportionment,'' in Proceedings
of the First ACM Workshop on Embedded Sensing Systems for Energy-Efficiency
in Buildings, New York, NY, USA, 2009, pp. 13--18.

Apportioning the total energy consumption of a building or organisation to
individual users may provide incentives to make reductions. We explore how
sensor systems installed in many buildings today can be used to apportion
energy consumption between users. We investigate the differences between a
number of possible policies to evaluate the case for apportionment based on
energy and usage data collected over the course of a year. We also study the
additional possibilities offered by more fine-grained data with reference to
case studies for specific shared resources, and discuss the potential and
challenges for future sensor systems in this area.

\begin{itemize}
\item If accounting data can be used to forecast and model future energy usage of an HPC center? so this can be communicated and be in
egrated with electricity grid.
\item If/how electricity grid providers can use energy and usage accounting data to plan electricity provisioning of an HPC center?
\item user-specific accounting data versus workload-specific accounting data.
\item accounting data in terms of HPC center components, cooling, systems, lighting, etc.
\end{itemize}
These are excellent questions. What you've outlined below is a set of value
of real-time data (the term "accounting" confused me earlier) of energy and
utilization for HPC systems. Some of these values are for EE and the rest is
how the electric grid service providers can benefit from it. For example,
telemetry data for wholesale DR markets and M{\&}V.

M{\&}V or Measurement and Verification refers to quantification of load shed
that a particular load is participating in. Typically, there are many
baseline methodologies that the utilities and ISOs use to calculate the
amount of DR a particular load/facility is providing through real-time and
day-ahead metered data. The metering and telemetry to provide the M{\&}V is
key in determining if a particular resource can participate in a DR market

and validate its performance for settlement (economics).

