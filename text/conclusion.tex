This paper explores the possibility of a new relationship between ESPs
and SCs with increased communication and engagement from both parties.

Because SCs have an increasingly large and fluctuating
power demand, they challenge their providers to supply a reliable
source of electricity.
ESPs are interested in partnering with customers,
like SCs, to create a more dynamic and resilient grid
by obtaining predictable demand forecasts and engaging in programs like 
demand response.

We focused our attention on the largest SCs in the United States. The two SCs with the largest electricity demand, ORNL and LLNL, have had very different experiences. 
ORNL's experience is that its electricity demand and fluctuations are not significant factors for their 
ESP. 
LLNL's experience is opposite to that of ORNL. Because of large swings in 
power usage, the LLNL SC was approached by their ESP with a 
request for daily predictable demand forecasts. That request began an ongoing relationship. 

The LANL SC's experience is similar to that of LLNL. SDSC has an even tighter relationship 
with their ESP, but this relationship involves the entire campus and not just the 
SC. 

As previous research with datacenters has shown \cite{LBNL-6560E}, SCs can serve as  
resources to the grid. To enable this, automation technologies and data communication standards, 
which can link the SCs with the electric grid and on-site power management strategies 
for grid services will play a key role to ease adoption and lower the participation costs. Power capping, 
shutdown, and job scheduling are identified as the most interesting management strategies with the highest 
leverage for responding to requests from ESPs. 

Nonetheless, the business case for the grid integration of SCs remains to be demonstrated. 
SCs have concerns that deploying these strategies might have an adverse impact on 
their primary mission. One of the key enablers for SCs to participate in 
electricity markets (for example, demand response, electricity prices) is having markets that value their 
participation. In other areas like commercial buildings and select industrial facilities, benefits to 
both ESPs and customers are well documented. However, as the electrical grid 
and new dynamic loads such as SCs evolve, the markets need mechanisms to identify 
and provide value of participation (for example, cost, energy, carbon).

We are planning to pursue several areas in our future work.

We are planning a similar survey for Europe to explore if there is a more compelling business case 
in other geographies.
We expect the business value of such grid integration to be enhanced where the price of 
electricity is expensive, varies dynamically, or where there is strong reliance on expensive back-up 
generation (for example, India).  

We plan on following-up with the ESPs that support these US-based SCs. We note that this work's focus was from the perspective of the SC, and 
we are interested in hearing from the ESPs about what makes a customer more or less interesting or 
challenging with respect to grid integration.

With increasing variable renewable generation and price-based DR programs, the intra-hour fluctuations 
and demand forecasting are becoming increasingly important.
Electrical grid programs may react in different ways to the timescale of a SC's load response.
What are the trends 
in inter-hour fluctuation patterns? Is this a new behavior, an interim one, or one that is likely to get worse?
