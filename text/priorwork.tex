We now describe prior work done in Power and Thermal Management, Job Scheduling, and Load Migration in the HPC and data center communities. Software and hardware techniques to save energy and power have been studied extensively. However, most of this work does not take into consideration power management (especially cooling systems and IT equipment) in response to a request from an electrical service provider \cite {Ghatikar2012a}. 

\subsection{Power Management}
DVFS and power capping are two popular ways to manage node power. Prior work in the HPC domain looked at analytical models to understand energy consumption \cite{SpringerPPoPP2006,GeICPP2007,LiHPCA2006} and at trading execution time for lower power/energy \cite{CameronSC2005,HsuSC2005}. Several DVFS algorithms have also been proposed, such as CPUMiser \cite{GeICPP2007} and Jitter \cite{KappiahSC2005}. Varma et al, 2003 \cite{varma_control-theoretic_2003} demonstrated system-level DVFS techniques. They monitored CPU utilization at regular intervals and performed dynamic scaling based on their estimate of utilization for the next interval. Springer et al.~\cite{springer:06} analyzed HPC applications under an energy bound. Rountree et al. used linear programming to find near-optimal energy savings without degrading performance \cite{rountree:07} and implemented a runtime system based on this scheme \cite{rountree:09}. 

There also has been work in the real-time systems community to solve the DVFS scheduling problem using mixed integer linear programming on a single processor\cite{IshiharaISLPED1998,SaputraLCTES2002,SwaminathanRTSS2000,SwaminathanASPDAC2001}. Other real-time approaches looked at saving energy \cite{MoncusiRTSS2003,MochockiICCAD2002,MochockiRTAS2005,ZhuTPDS2003,ZhangDAC2002}. 

In addition, there has been active research in the domain of virtual machines. Von Laszewski et al. \cite{von_laszewski_power-aware_2009} presented an efficient scheduling algorithm to allocate virtual machines in a DVFS-enabled cluster by dynamically scaling the supplied voltages. Dhiman et al. designed vGreen \cite{dhiman_vgreen:_2009}, which is a system for energy efficient computing in 
virtualized environments. They linked online workload characteristics to dynamic VM scheduling decisions and achieved better performance, energy
efficiency and power balance in the system. Curtis-Maury et. al \cite{Curtis1,Curtis2,Curtis3} introduced Dynamic Concurrency Throttling, which is a technique to dynamically optimize for power and performance by varying the number of active threads in parallel codes. 

Chip power measurement and capping techniques were initially introduced with the Running Average Power Limit (RAPL) interface on Intel Sandy Bridge processors \cite{IntelSDM,David2010}. In the HPC domain, Rountree et al.~\cite{Rountree2012} proposed RAPL as an alternative to DVFS and analyzed application performance under hardware-enforced power bounds. They also established that variation in power directly translates to variation in application performance under a power bound. Patki et al. \cite{Patki1} used power capping techniques to demonstrate how hardware overprovisioning can improve HPC application performance under a global power bound significantly. Overprovisioning was also explored in the data center community \cite{femal:04}.

\subsection{Thermal Management}
Thermal and cooling metrics are becoming important in HPC resource management. Runtime cooling strategies are mostly
job-placement-centric. These techniques either aim to place incoming computationally intensive jobs in a thermal-aware manner on servers with
lower temperatures or attempt to migrate or load-balance jobs from high-temperature servers to servers with lower temperatures.

Kaushik et. al \cite{kaushik_t*:_2012} proposed \emph{T*}, a system that is aware of server thermal profiles and reliability as well as data semantics (computation job rates, job sizes, etc). This system saves cooling energy costs by using thermal-aware job placements without trading off performance. This paper assumes that the given grid is a constant. However, it is expected that future grid infrastructures will evolve due to grid integration solutions \cite{he_architecture_2008}. This may make thermal management more challenging.

Aikema et. al \cite{aikema_electrical_2011} explored the potential for HPC centers to adapt to dynamic electrical prices, to variation in carbon intensity within an electrical grid, and to availability of local renewables. Their simulations demonstrated that 10- 50 \% of electricty costs could potentially be saved. They also concluded that adapting to the variation in the electrical grid carbon intensity was difficult, and that adapting to local renewables could result in significantly higher cost savings. 

Sarood et. al \cite{SaroodSC11} designed a runtime system that does temperature-aware load balancing in data centers using DVFS and task migration. They also discussed how hotspots could be avoided in data centers, and showed cooling costs can be reduced by up to 48\% with temperature-aware load balancing.

\subsection{Job Scheduling}
The problem of scheduling jobs has been extensively studied. Most resource managers implement the First Come First Serve (FCFS) policy
as a simple but fair strategy for scheduling jobs. However, FCFS suffers from low system utilization. A common optimization is backfilling
\cite{lifka_anl/ibm_1995,mualem_utilization_2001,feitelson_parallel_2004}. Backfilling improves system utilization by scheduling jobs with small resource requests out of order on idle nodes.

In \cite{yang_integrating_2013} \cite{zhou_reducing_2013}, \textbf{job scheduling} as a DR strategy and
\textbf{dynamic pricing} as a grid integration program have been used to
propose \textbf{a power-aware job scheduling} approach to reduce
\textbf{electricity costs} \textit{without degrading system utilization}. 
The novelty of the proposed job scheduling
mechanism is its ability to take \textit{the variation of 
the price of electricity }into consideration as a means to make
better decisions of the timing of scheduling jobs with diverse power
profiles. Experiments on an IBM Blue Gene/P and a cluster system as
well as a case study on Argonne's 48-rack IBM Blue Gene/Q system have
demonstrated the effectiveness of this scheduling approach. Preliminary
results show a \textbf{23{\%}} reduction in the cost of electricity for HPC systems.

%Patki
%This reference is not *power capping*, it is power-aware scheduling in data centers. 
Fan et al. \cite{PowerAwareServer1} discussed power-aware job scheduling in the data center domain. 
They discussed implementation of power capping with a \textbf{power monitoring system} based on a power estimation
method or direct power sensing, and a \textbf{power throttling mechanism}. Power throttling generally works best when there is 
a set of jobs with loose service level guarantees or low priority that can be
forced to reduce consumption when the datacenter is approaching the power cap value. They suggested that power consumption 
can be reduced simply by de-scheduling tasks or by using any available component-level power management knobs.

Etinski et al. \cite{Etinski1,Etinski2,Etinski3,Etinski4} explored scheduling under a power budget in supercomputing and analyzed bounded slowdown of jobs. In their series of papers, they introduced three policies. Their first policy is based looks at current system utilization and uses DVFS during job launch time to meet a power bound. Their second policy meets a bounded slowdown condition without exceeding a job-level power budget. Their third policy improves upon the former by analyzing job wait times and adding a reservation condition. 

A grid computing infrastructure with large amount of computations normally
contains parallel machines (a supercomputer cluster) as main computational
resources.  \cite{foster_anatomy_2001} [Fos01]
Incoming jobs to Grid's local resources are scheduled by
local scheduling system. Local scheduling system for parallel machines
typically use batch queued space-sharing and its variants as scheduling
policies. Most current local schedulers use backfilling strategies with FCFS
queue-priority order as policy for parallel job scheduling.

There are many use cases in a grid computing environment that require QoS
guarantees in terms of guaranteed response time, including time-critical
tasks that must meet a deadline, which would be impossible without a start
time guarantee. Furthermore, providing a time guarantee enables the job to be
coordinated with other activities, essential for co-allocation and workflow
applications. Advance reservation is a guarantee for the availability of a
certain amount of resources to users and applications at specific times in
the future 
\cite{foster_distributed_1999} [Fos99]. The advance reservation feature requires local scheduling
systems to support a reservation capability beside a batch-queued policy for
local and normal jobs. In load migration, we encounter the need to deliver
resources at specific times in order to accept jobs from other HPC centers
to respond to their demand enforced by the electricity grid. This requirement
can be achieved by advance reservations 
\cite{foster_distributed_1999}
[Fos99]. Modern resource management
and scheduling systems such as Sun Grid Engine, PBS, OpenPBS, Torque, Maui,
and Moab support backfilling and advance reservation capabilities.

\subsection{Load Migration}
Chiu et. al \cite{chiu_electric_2012} discussed a electrical grid balancing problem that was experienced in the Pacific Northwest. In order to match electricity supply and balance the electrical grid, they proposed low-cost geographic load migration. They also suggested that a symbiotic relationship between datacenters and electrical grid operators that leads to mutual cost benefits could work well.  Ganti et al. \cite{Ghatikar2012b} looked at two applied cases for distributed data centers. The results show that load migration is possible in both homogenous and heterogeneous systems. Their migration strategies were based on a manual process and can benefit from automation.
