\noindent
\textbf{What is your ``total facility energy''?}
This should be the same as the total facility energy number that is used for calculating PUE, 
as defined by the Green Grid Whitepaper \#49.  Total facility energy includes all 
HPC equipment energy plus everything that supports the HPC equipment using energy, such as:
\begin{itemize} [nosep]
\item[{-}]
Power delivery components, including UPS systems, switchgear, generators, batteries, power 
distribution units (PDUs), and distribution losses external to the IT equipment  
\item[{-}]
Cooling system components, such as chillers, cooling towers, pumps, computer 
room air conditioning units (CRACs), computer room air handling units (CRAHs), and direct 
expansion air handler (DX) units  
\item[{-}]
Other miscellaneous component loads, such as data center lighting.
Ideally, total facility energy is monitored over a period of one year, taking 
ongoing measurements to compensate for peak and nominal loading changes that occur within the data 
center. If it is not possible to monitor energy consumption over a full year, select a period 
of time not less than one month and verify that the loading within the data center 
during that time is typical.	
\end{itemize}

\wl
\noindent
\textbf{What is your total HPC load?}
Note:  This should be the same number that you use when calculating PUE, as defined by 
The Green Grid WP\#49.  Again, this is for the facility that houses your Top100-sized system.  	
Please provide exact number in comment box, if possible.	

\wl
\noindent
\textbf{What is your facility PUE?}
Please provide exact number in comment box, if possible.

\wl
\noindent
\textbf{What is your facility's theoretical peak energy, as the infrastructure is currently fit up.}
Please provide exact number in comment box, if possible.	

\wl
\noindent
\textbf{What is the maximum variation in total facility energy that is likely to re-occur?}
Please provide exact number in comment box, if possible.

\wl
\noindent
\textbf{How often does this variation occur?}
If there is any regular pattern to this variation, please describe the circumstances.  
Include the reason for the variation, the magnitude and duration if possible.  
E.g., ``There is a 5MW drop every two weeks for a 6 hour period during Preventative Maintenance 
periods.'' 

\wl
\noindent
\textbf{COARSE GRAINED POWER MANAGEMENT:}  manage power for the HPC system or sub-system 
(could include storage, networking as well as compute sub-systems). 
Example: power capping. 

\wl
\noindent
\textbf{FINE GRAINED POWER MANAGEMENT:}  intelligent built-in power management.  
Examples: voltage and frequency governors, hibernation.

\wl
\noindent
\textbf{LOAD MIGRATION:}  shift computing loads to a different electrical grid.   

\wl
\noindent
\textbf{JOB SCHEDULING:}  Job shifting or queuing (scheduling) has historically been used as a 
strategy for managing CPU utilization, but could also be used to manage the energy 
outilization of IT equipment. 

\wl
\noindent
\textbf{BACK-UP SCHEDULING:}  Defer data storage processes to off-peak periods  
Other (please specify).	

\wl
\noindent
\textbf{SHUTDOWN:} Graceful shutdown of idle HPC equipment loads. Usually applies when there is redundancy 

\wl
\noindent
\textbf{LIGHTING CONTROL:} With advance warning, data center lights could be shutdown completely. 

\wl
\noindent
\textbf{TEMPERATURE ADJUSTMENT:} widen acceptable (ASHRAE Thermal Conditions) temperature setpoint 
ranges and humidity levels for short periods.  

\wl
\noindent
\textbf{BACK-UP RESOURCES:}  Using generators and other electrical storage devices. 
Other (please specify)	.

\wl
\noindent
\textbf{Are there any other strategies that you use to manage and control your total facility 
energy in response to a request from your energy utility/provider?}
Please describe. 

\wl
\noindent
\textbf{Please evaluate as high, medium or low the MW impact of each of these strategies as a 
response to a grid request. }
\begin{itemize} [nosep]
\item[{-}] Power capping	
\item[{-}] Load migrations	
\item[{-}] Temperature adjustments	
\item[{-}] Clock speeds	
\item[{-}] Lighting control	
\item[{-}] Job scheduling	
\item[{-}] Back-up scheduling	
\item[{-}] Idle management	
\item[{-}] Shutdown	
\item[{-}] Back-up resources	
\end{itemize}

\wl
\noindent
\textbf{PEAK SHEDDING:}  Utility provider arrangements used to reduce peak load, where the 
reduced load is not shifted to another time. 
Describe the nature of the communication.

\wl
\noindent
\textbf{PEAK SHIFTING:}  Utility provider arrangements where the load during peak times is moved, typically 
to non-peak hours. 
Describe the nature of the communication.	

\wl
\noindent
\textbf{DYNAMIC PRICING:}  Time varying pricing arrangements used to increase, shed or shift electricity 
consumption. There are two types of pricing, peak and real-time.  Peak pricing is pre-scheduled; 
however, the consumer does not know if a certain day will be a peak or a non-peak day until day-ahead 
or day-of.  Real-time pricing is not pre-scheduled; prices can be set day-ahead or day-of.	
Describe the nature of the communication.	

\wl
\noindent
\textbf{GRID SCALE STORAGE:}  Methods used to store electricity on a large scale. Pumped-storage 
hydroelectricity is the largest-capacity form of grid energy storage. 
Describe the nature of the communication.	

\wl
\noindent
\textbf{RENEWABLES:}  Variability in the electric power generation from renewable resources 
and the methods used to respond to that variability. 
Describe the nature of the communication.	

\wl
\noindent
\textbf{FREQUENCY RESPONSE:}  Methods used to keep grid frequency constant and in-balance. 
Generators are typically used for frequency response, but any appliance that operates 
to a duty cycle (such as air conditioners and heat pumps) could be used to provide 
a constant and reliable grid balancing service by timing their duty cycles in response 
to system load.   	
Describe the nature of the communication.	

\wl
\noindent
\textbf{REGULATION (Up or Down):} Methods used to maintain that portion of electricity generation 
reserves that are needed to balance generation and demand at all times.  Raising supply 
is up regulation and lowering supply is down regulation. There are many types of reserves 
(e.g., operating, congestion), distinguished by who controls them and what they are used for. 
Describe the nature of the communication.	

\wl
\noindent
\textbf{CONGESTION:} Methods used to resolve congestion that occurs when there is not enough 
transmission capability to support all requests for transmission services. Transmission 
system operators must re-dispatch generation or, in the limit, deny some of these 
requests to prevent transmission lines from becoming overloaded.  Or, methods used to 
resolve congestion that occurs when the distribution control system is overloaded.  It 
generally results in deliveries that are held up or delayed.     
Describe the nature of the communication.

\wl
\noindent
\textbf{Is there information you would like from your provider that you are not getting?}
If yes, please describe what you would like to know.	
Describe what you would like to know	

\wl
\noindent
\textbf{Is your provider asking for information from you that you are not able to provide?}
If yes, please describe what they are asking for.	

\wl
\noindent
\textbf{Do you experience any power quality issues at your HPC facility?}
If yes, please describe. 


