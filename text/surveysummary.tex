For the purposes of this paper, this appendix contains a summary of the questionnaire.
%The complete text of our questionnaire, which includes some background 
%and explanatory material, is available at
%\href{https://www.surveymonkey.com/s.aspx?sm=Y%2feFYAHaRP1tw5bhUMdtfg%3d%3d}{EEHPCWG questionnaire}.
%Note that the questionnaire at that site is no longer active; any responses are not saved.

The questionnaire is divided into the following three sections: 
\begin{itemize}

\item
Facility Energy. The total facility energy and the total HPC load should be the 
same number that you use when 
calculating PUE, as defined by the Green Grid Whitepaper \#49.

\item
Management and Control. 
Please answer whether or not you employ any of the strategies described below for managing 
and controlling total facility energy in response to a request from your Electrical Utility/Provider.
You may use some of these same strategies for improving energy efficiency. 
Answer ``Yes'' only when the strategy is used at least in part for grid response. 
Answer ``No'' if the strategy is only used for improving energy efficiency.

\item
Electrical/Utility Provider Information.
Answers to these questions help us understand the nature of any relationship you might have 
between your HPC facility and your site's electric utility/provider.
Please answer ``Yes'' if you have had any communication about the following programs and 
methods with your site's electric utility/provider.
For each program and/or method for which there has been communication, please describe 
the nature of that communication in the comments.
\end{itemize}

\subsection*{Facility Energy}
\begin{enumerate}  %[nosep]
\item
What is your ``total facility energy?''

\item
What is your total HPC load?

\item
What is your facility PUE?

\item
What is your facility's theoretical peak energy, as the infrastructure is currently fit up.

\item
What is the maximum variation in total facility energy that is likely to re-occur? 

\item
 How often does this variation occur?

\item
If there is any regular pattern to this variation, please describe the circumstances. 
Include the reason for the variation, the magnitude and duration if possible. 
For example, ``There is a 5MW drop every two weeks for a 6 hour period during Preventative 
Maintenance periods.''
\end{enumerate}

\subsection*{Management and Control}

\begin {enumerate} %[nosep]
\setcounter{enumi}{7}
\item
COARSE-GRAINED POWER MANAGEMENT: manage power for the HPC system or subsystem 
(could include storage, networking as well as compute sub-systems). Example: power capping.

\item
FINE-GRAINED POWER MANAGEMENT: intelligent built-in power management. 
Examples: voltage and frequency governors, hibernation.

\item
LOAD MIGRATION: shift computing loads to a different electrical grid. 

\item
JOB SCHEDULING: Job shifting or queuing (scheduling) has historically been used 
as a strategy for managing CPU utilization, but could also be used to manage the 
energy utilization of IT equipment.

\item
BACK-UP SCHEDULING: Defer data storage processes to off-peak periods 

\item
SHUTDOWN: Graceful shutdown of idle HPC equipment loads. Usually applies when there is redundancy

\item
LIGHTING CONTROL: With advance warning, datacenter lights could be shutdown completely.

\item
TEMPERATURE ADJUSTMENT: Widen acceptable (ASHRAE Thermal Conditions) temperature setpoint 
ranges and humidity levels for short periods. 

\item
BACK-UP RESOURCES: Using generators and other electrical storage devices.

\item
Are there any other strategies that you use to manage and control your total 
facility energy in response to a request from your energy/utility provider.
Please describe.

\item
Please evaluate as high, medium or low the MW impact of each of these strategies 
as a response to a grid request. 

\begin{itemize} %[nosep]
\item[{-}]
Power capping 
\item[{-}]
Load migrations	
\item[{-}]
Temperature adjustments	
\item[{-}]
Clock speeds	
\item[{-}]
Lighting control	
\item[{-}]
Job scheduling	
\item[{-}]
Back-up scheduling	
\item[{-}]
Idle management	
\item[{-}]
Shutdown	
\item[{-}]
Back-up resources	
\end{itemize}
\end{enumerate}

\subsection*{Electrical Utility/Provider Information}

\begin{enumerate} %[nosep]
\setcounter{enumi}{18}
\item
PEAK SHEDDING: Utility provider arrangements used to reduce peak load, where the reduced 
load is not shifted to another time.

\item
PEAK SHIFTING: Utility provider arrangements where the load during peak times is moved, 
typically to non-peak hours.

\item
DYNAMIC PRICING: Time varying pricing arrangements used to increase, shed or shift 
electricity consumption. There are two types of pricing, peak and real-time. 
Peak pricing is pre-scheduled; however, the consumer does not know if a certain day will be 
a peak or a non-peak day until day-ahead or day-of. Real-time pricing is not pre-scheduled; 
prices can be set day-ahead or day-of.

\item
GRID SCALE STORAGE: Methods used to store electricity on a large scale. Pumped-storage 
hydroelectricity is the largest-capacity form of grid energy storage.

\item
RENEWABLES: Variability in the electric power generation from renewable resources and 
the methods used to respond to that variability.

\item
FREQUENCY RESPONSE: Methods used to keep grid frequency constant and in-balance. 
Generators are typically used for frequency response, but any appliance that operates to 
a duty cycle (such as air conditioners and heat pumps) could be used to provide a constant 
and reliable grid balancing service by timing their duty cycles in response to system load. 

\item
REGULATION (Up or Down): Methods used to maintain that portion of electricity generation 
reserves that are needed to balance generation and demand at all times. Raising supply is 
up regulation and lowering supply is down regulation. There are many types of 
reserves (e.g., operating, congestion), distinguished by who controls them and what they are used for.

\item
CONGESTION: Methods used to resolve congestion that occurs when there is not enough 
transmission capability to support all requests for transmission services. Transmission 
system operators must re-dispatch generation or, in the limit, deny some of these 
requests to prevent transmission lines from becoming overloaded. Or, methods used to resolve 
congestion that occurs when the distribution control system is overloaded. It generally 
results in deliveries that are held up or delayed. 

\item
Is there information you would like from your provider that you are not getting? If yes, 
please describe what you would like to know.

\item
Is your provider asking for information from you that you are not able to provide? If yes, 
please describe what they are asking for.

\item
Do you experience any power quality issues at your HPC facility? If yes, please describe.

\end{enumerate}
