%The EE HPC WG took as their starting point a
%model developed by LBNL's Demand Response Research
%Center. This model describes strategies that datacenters might employ for utility programs to manage their electricity and
%power requirements to lower costs and benefit from utility
%incentives. The EE HPC WG adopted this model %with slight tweaks  TP:This makes the model seem trivial, which it is not.
%to reflect the supercomputing environment
%focus (as opposed to %TP: Replace 'versus' with 'as opposed to'
%the datacenter focus described by LBNL`s Demand
%Response Research Center).
%%TP: Rephrasing the following sentence
%%For purposes of this paper, we define SCs 
%%as distinct from datacenters as having 
%%significantly higher system utilization and thus little or no 
%%virtualization.  
%
%It is important to highlight the differences between SCs and datacenters. Unlike datacenters, SCs are more performance oriented, have significantly higher system utilization, and use little or no virtualization. 
%Additionally, supercomputing
%applications are distinguished by their lack of geographical
%portability due to security concerns, data size and machine-specific
%optimizations.  
%
%We also note that SCs tend to be more
%energy efficient than datacenters. Power Usage Effectiveness (PUE) is a good measure for energy efficiency. PUE is the ratio of the total energy supplied for the facility to the amount of energy that actually reaches the IT infrastructure. A PUE of $1.0$ is ideal. In our survey, none of the SCs exceeded a Power Usage Effectiveness (PUE) of $1.53$, while the average PUE for a datacenter falls in the range of $1.91$ and $2.9$ ~\cite{Niccolai}.
%
%\subsection{Electricity Service Provider Methods and \\Programs}
%%TP: Change title slightly
\label{sub:EPP}
An ESP seeks to provide efficient and reliable generation, transmission, and 
distribution of electricity. \emph{Methods} and \emph{Programs} employed by ESPs and their consumers 
are key to managing and balancing the supply and demand of electricity. While the \textit{methods} 
describe how ESPs manage supply, the \textit{programs} describe the activities that 
the ESPs can offer to their consumers in order to balance demand with supply.

Although critical to ESPs, methods are generally not visible to the consumer of the electricity because they
operate within the generation or transmission stations. These methods are the major means by which supply and demand of electricity are managed.

ESP programs encourage customer responses to target both energy efficiency and real-time management of demand for electricity. Real-time programs can be \emph{day-ahead} or \emph{day-of}. Day-ahead programs refer to timescales of notification and responses from the customer that are determined based on advanced forecasting and capacity planning (for example, day-ahead 24-hour wholesale electricity prices). The programs are day-of are the ones when the notification and responses are based on same-day capacity planning or emergency response. 

An example of an ESP program 
that encourages energy efficiency would be to provide home consumers a financial incentive for replacing 
single pane windows with double pane windows. On the other hand, an example that illustrates programs that help 
with real-time demand management would be to provide a financial incentive for reducing load 
during high demand periods (such as hot summer afternoons when air conditioners are heavily utilized). 

The following is a list and brief definitions of key methods and programs.  

\subsection{Methods}
\begin{itemize}
\item Regulation (Up or Down): Methods used to maintain the portion of electricity generation reserves 
that are needed to balance generation and demand at all times. Raising supply is \emph{up} regulation and lowering 
supply is \emph{down} regulation. There are many types of reserves (for example, operating reserves, ancillary services), distinguished by who manages them and what they are used for.

\item Transmission Congestion: Methods used to resolve congestion that occurs when there is not enough 
transmission capability to support all requests for transmission services. Transmission system operators 
must re-dispatch generation or, 
in the limit, deny some of these requests to prevent transmission lines from becoming overloaded.

\item Distribution Congestion:  Methods used to resolve congestion that occurs when the 
distribution control system 
is overloaded.  It generally results in deliveries that are held up or delayed.  

\item Frequency response:  Methods used to keep grid frequency constant and in-balance. 
Generators are typically used for frequency response, but any appliance that operates to a duty cycle 
(such as air conditioners and heat pumps) could be used to provide a 
constant and reliable grid balancing service by timing their duty cycles in response to system load.   

\item Grid Scale Storage:  Methods used to store electricity on a large scale. 
Pumped-storage hydroelectricity is the largest-capacity form of grid energy storage. 

\item Renewables:  Methods used to manage the variable uncertain generation nature of 
many renewable resources. 
\end{itemize}

\subsection{Programs}
\begin{itemize}
\item Energy Efficiency:  Programs used to reduce overall electricity consumption.

\item Peak Shedding:  Programs used to reduce load during peak times, 
where the reduced load is not used at a later time. 

\item Peak Shifting:  Programs where the load during peak times is moved to, typically, non-peak hours. 

\item Dynamic Pricing:  Time varying pricing programs used to increase, shed,
 or shift electricity consumption. 
The two types of pricing are peak and real-time.  Peak pricing is pre-scheduled; however, the consumer 
does not know if a certain day will be a peak or a non-peak day until day-ahead or day-of.  
Real-time pricing is not pre-scheduled; prices can be set day-ahead or day-of.
\end{itemize}

Although these methods and programs have historically not been relevant to SCs,
the following example illustrates their potential relevance.
The generation capacity requirements and response timescales vary across the country for ESPs and operators. For example, the New England independent system operator (ISO-NE) uses a method 
of regulation and reserves that relies heavily on a day-ahead market program. This provides an opportunity 
for demand side resources---such as SCs with renewable energy sources---to participate in the 
market supplying the ISO-NE with electricity. It also makes the ISO-NE particularly sensitive to major 
fluctuations in electricity demand, which, as discussed further in the questionnaire section, is an emerging 
characteristic of some of the largest SCs.%\footnote {http://drrc.lbl.gov/sites/drrc.lbl.gov/files/LBNL-5958E.pdf}

This paper assumes that the given grid is a constant. However, it is expected 
that future grid infrastructures will 
evolve with smart-grid capabilities. 
