In November 2004, the Blue Gene/L system at Lawrence Livermore National Laboratory
became the fastest computer in the Top 500~\cite{Top500}, displacing the NEC Earth Simulator,
the previous champion. This change marked the transition from supercomputing gains based
on ever-higher-performance components to systems that comprised of far larger numbers of 
slow but energy-efficient components. However, total system power consumption continued to rise,
and we are now poised to begin a second transition to ``power-limited computing'' and ``power-aware computing''. The new
model has been exemplified by the US Department of Energy issuing guidance that the first
DOE exascale machine should not exceed 20 MW; effectively a $1000x$ performance improvement
with only a $3x$ increase in power. 

However, the problem is not as simple as provisioning 20 MW. Ultimately, SCs optimize for
performance per dollar, not performance per Watt, and flexibility in power consumption
can be expected to result in lower overall prices. Use of green technologies such as
wind and solar energy may also lead to cheaper but less predictable sources of power.
In addition, as described in Section~\ref{sec:ESPintegration}, ESPs may request a change in timing and/or magnitude of demand by SCs. To adapt to this new landscape, SCs may employ one or more strategies to control their electricity demand. We describe some of these strategies below. 

\begin{itemize}
\item {\bf Fine-grained Power Management} refers to the ability to control SC system power 
and energy with tools that offer high resolution control and can target specific 
low level sub-systems. A typical example is CPU voltage and frequency scaling.

\item {\bf Coarse-grained Power Management} also refers to the ability to control SC 
system power and energy, but contrasts with fine-grained power management in 
that the resolution is low and it is generally done at a more aggregated level. 
A typical example is power capping.

\item {\bf Load Migration} refers to temporarily shifting computing loads from 
an SC system in one site to a system in another location that has stable power supply. 
This strategy can also be used in response to change in electricity prices.

\item {\bf Job Scheduling} refers to the ability to control SC system power 
by understanding the power profile of applications and queuing the 
applications based on those profiles.

\item {\bf Back-up Scheduling} refers to deferring data storage processes to off-peak periods.

\item {\bf Shutdown} refers to a graceful shutdown of idle SC equipment. It usually 
applies when there is redundancy.

\item {\bf Lighting Control} allows for datacenter lights to be shutdown completely.

\item {\bf Thermal Management} is widening temperature set-point ranges and 
humidity levels for short periods.
\end{itemize}

These strategies can be used temporarily to modify loads in response to a request from an ESP. Additionally, some of these strategies could eventually be used at all times to improve overall energy efficiency if the SC sees no operational issues. Two examples may help to clarify this distinction. Temporary load migration is an example of a strategy that is well suited to responding to an ESP request, but is not likely to improve energy efficiency (lowering aggregate energy use). Fine-grained power management, on the other hand, can be used at all times and is more likely to be used for improving overall energy efficiency, unless the strategy is specifically used in response to an ESP's request. 

SC system power management has a very broad range of implementations and warrants greater exploration. For example, the coarse-grained and fine-grained strategies described above can be implemented at many levels of the system hierarchy---from node-level to site-level. We discuss these implementation approaches below.  

\begin{itemize}
\item \textbf{Node level:} Controlling power ultimately requires control of individual
components. Historically, this control has been accomplished through Dynamic Voltage/Frequency Scaling (DVFS), which allows the processor to use a lower voltage at the cost of a slower clock frequency. Newer technologies such as Intel's Running Average Power Limit leverage DVFS to guarantee that a user-specified processor power bound will, on average, not be exceeded over the duration of a short time window. DVFS can also be found on accelerator components such as NVIDIA's Kepler GPGPU. Other efforts reduce DRAM power by optimizing reads and writes, thus allowing the memory to spend more time in a lower-power state. Several processor configuration options have indirect but significant effects on power consumption. For example, the choice of the number of cores to use, whether or not to enable hyperthreading, and the use of ``turbo'' modes will change the power/performance curve.

\item \textbf{Job level:} Each of the node-level controls requires a tradeoff between
power and performance. SC resources are typically oversubscribed, so degrading performance
to save power and energy ultimately results in less science getting done. However, at the 
job level, load imbalance provides opportunities to slow nodes that are off of the critical
path of execution without slowing the overall job execution time. Traditionally, load 
rebalancing strategies have focused on moving bytes around the job allocation. With 
power control, we can now rebalance power as well as work.

\item \textbf{System level:} While most SCs use time and space partitioning (where a node
only runs a single job at a time), there are still shared resources that must be managed
across jobs. Periodic checkpointing saves sufficient job state to a filesystem shared 
across jobs so that a job may be restarted from a recent point in case a fault occurs.
Because these checkpoints involve much more data motion than normal execution, power 
spikes can be observed at the node level (particularly DRAM), network, and filesystem.
These checkpoints may need to be coordinated across large jobs to prevent unnecessary
performance degradation.
%TP: Checkpointing results in power spikes because more data motion takes place -- you need to save the state of the application (memory) to the filesystem. CPR in SCs is hierarchical, there's local storage first, and then this is written to a central (or a few central) file system serves via the network. This may lead to congestion (leading to more network power) as well as more power being used on the node.

\item \textbf{Scheduler level:} Up through the system level, power control is evaluated
using the execution time of individual jobs. The scheduler optimizes for overall throughput
rather than individual job performance. At this point, scheduling is a two-dimensional 
problem: jobs request a certain number of nodes for a certain duration. As power-limited
computing becomes more common, schedulers will add power bounds to this mix: a job will
be allowed nodes, time, and a certain number of watts (the responsibility for not exceeding
the job power bound rests with the system software, not the user or application). The 
scheduler not only determines when jobs in the queue begin execution, but also what happens
when a job exits the system. Depending on the priorities of already-running jobs and the
priorities of jobs in the queue, the best solution in terms of throughput may be to idle
the recently-freed nodes and redistribute the freed power to running jobs.

\item \textbf{Site level:} At the level of the machine room (or multiple machine rooms),
decisions must be made as to how much power should be allocated for cooling versus computation, which requires understanding how temperature interacts with performance. A higher intake air temperature uses less cooling power but results in higher static processor power and may limit opportunities for ``turbo'' mode in processors where it is available. As cooling power varies with outside air temperature, a single machine room temperature setpoint may not be the optimal solution in terms of overall performance. 
\end{itemize}



\subsection{Prior Work}
\label{sub:priorwork}
This paper pulls together several diverse research domains. In this section,
we provide an overview of prior work in these areas.

\subsubsection{Power Management}
Processor power management can be divided into two distinct eras. First, with 
the introduction of Dynamic Voltage Frequency Scaling, users were able to 
change the CPU clock speed of their processors, lowering both voltage and,
in most cases, energy: the program used less power and ran longer, but the
quadratic relation of power to frequency biased the results towards overall
energy savings. Early work included several modeling efforts focused on
the effects of CPU- and memory-boundedness on delay and energy in MPI programs~\cite{SpringerPPoPP2006,GeICPP2007,LiHPCA2006,CameronSC2005,HsuSC2005}. 
This work led to the CPUMiser~\cite{GeICPP2007} and Jitter runtime systems, which were designed 
to maximize energy saving consistent with a user-specified delay~\cite{KappiahSC2005}.
Treating energy savings as an optimization problem led to a linear programming
solution~\cite{rountree:07}. The follow-on Adagio runtime system slowed only
computation that could be proven to be off the critical path, leading to significant
energy savings with only negligible slowdown~\cite{rountree:09}. These 
techniques were also applied to non-MPI datacenter workloads~\cite{femal:04}. 

Other power saving approaches were attempted that did not use DVFS, but most
were not deemed relevant to the supercomputing environment. A notable exception
is Dynamic Concurrency Throttling, where energy savings are realized by varying
the number of threads at runtime~\cite{Curtis1,Curtis2,Curtis3}.

The research landscape changed considerably with the introduction of Intel's
Sandy Bridge processor. Turbo mode allowed higher clock frequencies to be
reached so long as fewer cores were in use, making for a nontrivial power- performance tradeoff
calculation. The Running Average Power Limit (RAPL) technology provided an onboard power model that allowed the processor to both
estimate power and, using rapid dithering of CPU clock frequencies, enforce a
user-specified power bound across a short time window~\cite{IntelSDM,David2010}. 
For the first time, users were able to ask questions about performance under 
power bounds. This new capability arrived concurrent with Department of Energy
guideline that exascale machines would be subject to power (as opposed to energy) bounds.

Initial work showed that while processor performance at a fixed frequency 
was reproducible across processors, execution in turbo mode or under a power
bound revealed significant performance variation~\cite{Rountree2012}. Further
work demonstrated a $2x$ performance improvement between conservative and
optimal processor configurations while executing under a power bound~\cite{Patki1}.

\subsubsection{Thermal Management}

Thermal management is a key driver for improving energy efficiency of datacenters as well as SCs. There are many strategies for thermal
management that can improve energy efficiency, such as free cooling and proper
airflow. This paper discusses two thermal management strategies that have an
opportunity for grid integration. The first strategy is controlling the inlet
temperature to the computing equipment, raising it as high as possible without
causing reliability induced hardware failures. The second strategy is using
thermally aware job scheduling.

In 2011, the American Society of Heating, Refrigeration and Air Conditioning
(ASHRAE) datacenter Technical Committee TC9.9 published guidelines that
expanded the environmental range for datacenters and SCs~\cite{ASHRAE:tgfdpe}. The environmental range includes factors such as temperature, humidity
and dew point and allowable rate of change. This expansion allows for
maintaining high reliability while achieving gains in energy efficiency. These
guidelines continue to be updated and the range continues to expand as the
industry collects more historical data showing trade-offs between reliability
and environmental factors.

It is implicit in the ASHRAE guidelines that a SC might be
able to increase temperature as a response to a request from an ESP. The guideline defines both recommended and
allowable environmental ranges. It also specifies a maximum rate of
change, which is most stringent for tape drives. For SCs, the
difference between the maximum recommended and allowable dry bulb temperature is
a minimum of 9 degrees F. The rate of change for tape drives is 9 degrees F per
hour (36 degrees F for solid state computing systems). Therefore, assuming
that SCs normally operate within the recommended range and
that they are willing to operate on occasion in the allowable range (or beyond),
it is theoretically possible to stay within ASHRAE thermal guidelines and use
temperature excursion as a grid-integration strategy. 

ASHRAE has also published a guideline on liquid cooling environmental ranges. At this point, however, the guidelines do not document rate of
change for liquid temperature. Although it is not explored in this paper, it
may be possible to use increases in liquid cooling temperature as a
grid-integration strategy as well. 

Ghatikar et. al \cite{Ghatikar2012a} describe field studies on using thermal
management as a grid-integration strategy. They demonstrate increasing ``facility HVAC temperature set points in order to decrease HVAC power
demand” in two different field locations. There was only a small
electricity demand decrease demonstrated.

Runtime cooling strategies are mostly job-placement-centric. These techniques
either aim to place incoming computationally intensive jobs in a thermal-aware
manner on servers with lower temperatures or attempt to migrate or load-balance
jobs from high-temperature servers to servers with lower temperatures.

Kaushik et. al \cite{kaushik_t*:_2012} proposed \emph{T*}, a system that is
aware of server thermal profiles and reliability as well as data semantics
(computation job rates, job sizes, etc). This system saves cooling energy costs
by using thermal-aware job placements without trading off performance.

Sarood et. al \cite{SaroodSC11} designed a runtime system that does
temperature-aware load balancing in datacenters using DVFS and task migration.
They also discussed how hotspots could be avoided in datacenters, and showed
cooling costs can be reduced by up to 48\% with temperature-aware load
balancing.

\subsubsection{Job Scheduling}
The problem of scheduling jobs has been extensively studied. Most resource
managers implement the First Come First Serve (FCFS) policy as a simple but fair
strategy for scheduling jobs. However, FCFS suffers from low system utilization.
A common optimization is \emph{backfilling}~\cite{lifka_anl/ibm_1995,mualem_utilization_2001,feitelson_parallel_2004}.
Backfilling improves system utilization by executing jobs with small resource
requests out of order on idle nodes.

Fan et al. \cite{PowerAwareServer1} discussed power-aware job scheduling in the
datacenter domain. They discussed a power monitoring system that could use
power capping (based on a power estimation method such as RAPL or direct power
sensing) and a power throttling mechanism. Such as system works well when is a
set of jobs with loose service level guarantees or low priority that can be
forced to reduce consumption when the datacenter is approaching the power cap
value. Etinski et al. \cite{Etinski1,Etinski2,Etinski3,Etinski4} explored
scheduling under a power budget in supercomputing and analyzed bounded slowdown
of jobs. In their series of papers, they introduced three policies. Their first
policy is based looks at current system utilization and uses DVFS during job
launch time to meet a power bound. Their second policy meets a bounded slowdown
condition without exceeding a job-level power budget. Their third policy
improves upon the former by analyzing job wait times and adding a reservation
condition. 

There are many use cases in a grid computing environment that require QoS
guarantees in terms of guaranteed response time, including time-critical tasks
that must meet a deadline. Foster et. al~\cite{foster_distributed_1999,foster_anatomy_2001} proposed \emph{advance
reservations} to achieve time guarantees. Advance reservation is a guarantee for
the availability of a certain amount of resources to users and applications at
specific times in the future. The advance reservation feature requires
scheduling systems to support reservation capabilities in addition to
backfilling-based batch scheduling. Modern resource management systems such as
Sun Grid Engine, PBS, OpenPBS, Torque, SLURM, Maui, and Moab support advance
reservation capabilities.

\subsubsection{Load Migration}
Chiu et. al \cite{chiu_electric_2012} discussed a electrical grid balancing
problem that was experienced in the Pacific Northwest. In order to match
electricity supply and balance the electrical grid, they proposed low-cost
geographic load migration. They also suggested that a symbiotic relationship
between datacenters and electrical grid operators that leads to mutual cost
benefits could work well. Ganti et al. \cite{Ghatikar2012b} looked at two
applied cases for distributed datacenters. The results show that load migration
is possible in both homogenous and heterogeneous systems. Their migration
strategies were based on a manual process and can benefit from automation.

\subsubsection{Dynamic Pricing}
Aikema et. al \cite{aikema_electrical_2011} explored the potential for HPC
centers to adapt to dynamic electrical prices, to variation in carbon intensity
within an electrical grid, and to availability of local renewables. Their
simulations demonstrated that 10- 50\% of electricity costs could potentially
be saved. They also concluded that adapting to the variation in the electrical
grid carbon intensity was difficult, and that adapting to local renewables could
result in significantly higher cost savings.

Power-aware resource management without degrading utilization has been proposed
as a DR strategy to reduce electricity costs
\cite{yang_integrating_2013,zhou_reducing_2013}. The novelty of the proposed job
scheduling mechanism is its ability to take the variation in electricity price
(dynamic pricing) into consideration as a means to make better decisions about
job start times. Experiments on an IBM Blue Gene/P and a cluster system as well
as a case study on Argonne's 48-rack IBM Blue Gene/Q system have demonstrated
the effectiveness of this scheduling approach. Preliminary results show a 23\%
reduction in the cost of electricity for HPC systems.

