
The EE HPC WG Team took as their starting point a model developed by
Lawrence Berkeley National Laboratory's Demand Reponse Research Center
\href{http://drrc.lbl.gov/publications/demand-response-and-open-automated-demand-response-opportunities-data-centers}
{http://drrc.lbl.gov/publications}
 that describes ways in which data centers and electricity service
providers may interact. This model describes programs that are used by the
electricity service providers to encourage particular behaviors by their
customers and methods used to balance the grid supply and demand of
electricity. It also describes strategies that data centers might employ for
managing their electricity and power requirements. The EE HPC WG Team
adopted this model with slight tweaks to reflect the HPC environment (versus
the general data center).

\subsection{Electricity Provider Programs and Methods}
Programs are used by the electricity service providers to encourage
particular behaviors by their customers. Methods used to balance the grid
supply and demand of electricity.

Below is a list of programs and methods:


\begin{itemize}
\item Energy Efficiency: Programs used to reduce overall electricity consumption, generally but not always at times of
 peak demand.
\item Peak Shaving (shed): Programs used to reduce load during peak times, where the reduced load is not used at a 
later time.
\item Peak Shaving (shift): Programs where the load during peak times is moved to, typically, non-peak hours.
\item Dynamic Pricing: Time varying pricing programs used to increase, shed or shift electricity consumption.
\item Grid Scale Storage: Methods used to store electricity on a large scale. Pumped-storage hydroelectricity 
is the largest-capacity form of grid energy storage.
\item Renewable (off-site): Methods used to manage the variable uncertain generation nature of many renewable resources.
\item Frequency response: Methods used to keep grid frequency constant and in-balance. Generators are typically used for frequency response.
\item Regulation (Up or Down): Methods used to maintain that portion of electricity generation reserves 
that are needed to balance generation and demand at all times.
\item Congestion: Methods used to resolve congestion that occurs when there is not enough transmission capability to 
support all requests for transmission services. Or, methods used to resolve congestion that occurs when the distributio
\end{itemize}


\subsection{Supercomputer Center Strageties}

Another dimension of the model is a list of strategies that a supercomputer
site might use for managing power in response to a request from their
electric service provider.

Although these strategies can be used for managing power in response to a
request from an electric service provider, many of them could also be used
for improving energy efficiency. It is the former that is of primary
interest to this investigation. Two examples may help to clarify this
distinction. Load migration is an example of a strategy that is well suited
to responding to an electric service provider request, but is not likely to
improve energy efficiency. Fine grained power management, on the other hand,
is more likely to be used for improving energy efficiency than for
responding to electric service provider requests.

Below is a list of strategies:

\begin{itemize}
\item Fine grained power management refers to the ability to control HPC system power 
and energy with tools that are high resolution control and can target specific 
low level sub-systems. A typical example is voltage and frequency scaling of the CPU.

\item Course grained power management also refers to the ability to control HPC 
system power and energy, but contrasts with fine grained power management in 
that the resolution is low and it is generally done at a more aggregated level. 
A typical example is power capping.

\item Load migration refers to temporarily shifting computing loads from 
an HPC system in one site to a system in another location.

\item Job scheduling refers to the ability to control HPC system power 
by understanding the power profile of applications and queuing the 
applications based on those profiles.

\item Back-up scheduling refers to deferring data storage processes to off-peak periods.

\item Shutdown refers to a graceful shutdown of idle HPC equipment. It usually 
applies when there is redundancy.

\item Lighting control allows for data center lights to be shutdown completely.

\item Thermal management is widening temperature set-point ranges and 
humidity levels for short periods.
\end{itemize}


