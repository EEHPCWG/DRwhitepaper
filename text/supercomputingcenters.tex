The EE HPC WG Team took as their starting point a model developed by LBNL's 
Demand Reponse Research Center \footnote{LBNL
Data Center Grid Integration Activities: http://drrc.lbl.gov/projects/dc} that
describes ways in which data centers and electricity service providers may
interact and how this integration can advance new market opportunities. This
integration model describes programs that are used by the electricity service
providers to encourage particular behaviors by their customers and methods used
to balance the grid supply and demand of electricity. It also describes
strategies that data centers might employ for utility programs to manage their
electricity and power requirements and lower costs. The EE HPC WG Team adopted
this model with slight tweaks to reflect the HPC environment (versus the
general data center).

%\href{http://drrc.lbl.gov/publications/demand-response-and-open-automated-demand-response-opportunities-data-centers}
%{http://drrc.lbl.gov/publications}

\subsection{Electricity Demand and Supply Side Management}

Electricity providers use demand-side management for both energy efficiency and to balance the electricity supply. 
Demand side management programs are used to manage the consumption of energy on the consumer’s side of the meter.  
Demand side management can include energy conservation, energy efficiency and peak load management, and responses to 
changing supply conditions.  The focus for this paper is on programs that are targeted at load management. 
Supply side management methods are used to ensure the efficient generation, transmission and distribution of 
electricity.  These methods 
manage the supply of energy on the electricity provider’s side of the meter to meet the changing demand conditions. 

There are many different ways that the end-users of electricity can modify or change their electricity consumption. 
The following is a list and 
brief definitions of key supply-side programs and demand-side management strategies that HPC data centers can utilize:



\begin{itemize}
\item SUPPLY SIDE PROGRAMS
\item Peak Load Response: Programs designed for responses during peak hour “events” and focus on reducing peaks on 
forecasted high-system load days. 
\item Reliability Response: These programs are designed for fastest, shortest duration responses. 
Response is only required during power system and intra-hour variability “events” that typically cannot be forecasted.
\item Regulation Response (Up or Down): Programs designed to follow minute-to-minute commands from the grid to 
balance the aggregate system load and generation. These programs can be expanded to manage the variable and 
uncertain generation nature of many renewable resources, supplement the methods for grid scale storage, which is 
used to store electricity on a large scale to meet the system ramping (up or down) requirements. Pumped-storage 
hydroelectricity is currently one of the largest forms of grid energy storage. Such 
responses can also be used for frequency response, which are methods used to keep grid frequency constant 
and in-balance.
\item Congestion Response: Methods used to resolve congestion that occurs when there is not enough transmission 
capability to support all requests for transmission services.  Or, methods used to resolve congestion 
that occurs when the distribution control system is overloaded. Peak response can be expanded for 
congestion management.
\item Dynamic Pricing: Day-ahead and day-off pricing programs that are designed to change dynamically in response 
to system load and generation conditions.  
\item DEMAND SIDE MANAGEMENT STRATEGIES
\item Load Shed: Strategies used to reduce the load consumption during peak times, 
where the reduced load is not used at a later time.
\item Load Shift:  Strategies where the load during peak times is moved to, typically, non-peak hours.
\item Price Response:  Strategies for time varying and real-time pricing programs used to motivate modification 
to electricity consumption. 
\item Continuous Energy Management: Emerging strategies that modify the load consumption (increase or decrease) 
based on intra-hour changing supply conditions.
\end{itemize}

As the electric grid evolves to a more dynamic and distributed system, its integration with and responses from 
HPC centers can be used to support supply-side programs.

\subsection{Supercomputer Center Response Strategies}

One dimension of the model to response to participate in supply-side programs is a list of demand-side 
management strategies 
that a supercomputer site might use to manage power in response to a request from their electric service provider. 

Although these strategies can be used for managing power in response to a
request from an electric service provider, many of them could also be used
for improving energy efficiency. It is the former that is of primary
interest to this investigation. Two examples may help to clarify this
distinction. Load migration is an example of a strategy that is well suited
to responding to an electric service provider request, but is not likely to
improve energy efficiency. Fine grained power management, on the other hand,
is more likely to be used for improving energy efficiency than for
responding to electric service provider requests.

Below is a list of strategies:

\begin{itemize}
\item Fine grained power management refers to the ability to control HPC system power 
and energy with tools that are high resolution control and can target specific 
low level sub-systems. A typical example is voltage and frequency scaling of the CPU.

\item Coarse grained power management also refers to the ability to control HPC 
system power and energy, but contrasts with fine grained power management in 
that the resolution is low and it is generally done at a more aggregated level. 
A typical example is power capping.

\item Load migration refers to temporarily shifting computing loads from 
an HPC system in one site to a system in another location.

\item Job scheduling refers to the ability to control HPC system power 
by understanding the power profile of applications and queuing the 
applications based on those profiles.

\item Back-up scheduling refers to deferring data storage processes to off-peak periods.

\item Shutdown refers to a graceful shutdown of idle HPC equipment. It usually 
applies when there is redundancy.

\item Lighting control allows for data center lights to be shutdown completely.

\item Thermal management is widening temperature set-point ranges and 
humidity levels for short periods.
\end{itemize}