Supercomputing centers (SCs) with petascale
\footnote{Petascale computing refers to computing systems capable of at
least \(10^{15}\) floating point instructions per second(FLOPS).} 
systems for high-performance computing (HPC) have an outsized 
impact on their electricity service providers (ESPs), with peak 
power demands in excess of 20MW and instantaneous power fluctuations of up to 8MW. 
As the HPC community moves towards exascale computing,\footnote{Exascale computing refers to computing systems capable of at
least \(10^{18}\)FLOPS.} we anticipate a growing number of facilities will
be reaching or exceeding these service levels, with significant potential 
effect on electrical grid reliability.
To moderate this risk, in this paper we seek to understand how these anticipated
usage patterns can be integrated safely into the power grid with minimal cost 
and disruption.

Being a "good citizen" on the grid has several historical precedents.
For example, electrically-intensive industries such as aluminium smelters 
have received preferential pricing in return for predictable loads and
flexibility in reducing power during periods of high consumption.
SCs are already adopting these strategies.  
For example, Lawrence Livermore National Laboratory (LLNL)
reduces its power usage when temperatures exceed 
100 degrees and residential power usage in the area surges,
Other centers are exploring the benefits of predicting hour-ahead and 
day-ahead use in concert with their providers.
A mutual understanding of concerns between the SC and ESP can 
produce a symbiotic relation that goes beyond the current producer-consumer 
paradigm, paving the way for possible integration of HPC with the electrical grid.

The Energy Efficient HPC Working Group (\href {http://eehpcwg.lbl.gov/}{EE HPC WG})  
investigates opportunities for large supercomputing sites to  
integrate more closely with their electricity service providers.
We seek to understand the willingness of SCs to
cooperate with their ESPs, their expectations from their ESPs, and the feasible measures
that SCs could employ to help their ESPs.
To achieve our objectives we developed a questionnaire and distributed it to 
the Top100 supercomputer centers in the United States. 
%NEED TEASER RESULTS HERE!

This work leverages prior work on datacenter and grid integration opportunities
done by Lawrence Berkeley National Laboratory's (LBNL) Demand Response 
Research Center \textbf{(cite LBNL DR datacenter integration paper)}
that describes the challenges and opportunities for datacenters and electricity
service providers to interact and how this integration can advance 
new market 
opportunities\footnote{LBNL Data Center Grid Integration Activities: http://drrc.lbl.gov/projects/dc}.
This integration model describes programs that are used by some of the electricity 
service providers to encourage particular responses by their customers and methods 
used to balance the electrical grid supply and demand.
Perhaps one of the most straightforward ways that SCs can begin
the process of participating in demand response is by participating in efforts to
develop software
infrastructure to manage their electricity requirements in a tightly coupled manner 
with their ESPs, facilitating both energy efficiency and grid reliability.

The paper is organized as follows.
Sections~\ref{sec:ESPintegration} and~\ref{sec:supercomputercenters} of this paper
describe in greater detail the model for 
integrating supercomputer centers and the electrical grid.
Section~\ref{sub:priorwork}
reviews prior work on HPC center strategies. Section~\ref{sec:questionnaire}
provides the results of the questionnaire. 
Section~\ref{sec:opportunities} 
discusses the several opportunities, solutions and barriers that have been highlighted
by the survey results. We offer our conclusions in Section~\ref{sec:conclusion} along with our plan for future work.  Additional authors are listed in Section~\ref{sec:additional} and
the \nameref{Appendix} summarizes the survey questions.
