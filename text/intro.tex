Supercomputer centers with petascale systems for high-performance 
computing (HPC) are realizing the large impact they will be 
putting on their electricity service providers with peak power demands of 20MW and instantaneous power fluctuations of 8MW. 
 
The Energy Efficient HPC Working Group 
(\href {http://eehpcwg.lbl.gov/}{EE HPC WG}) 
has been investigating opportunities for large supercomputing sites to more closely 
integrate with their electricity service providers. This paper documents the 
results of this investigative activity. 

Leveraging prior work on data center and grid integration opportunities done
by Lawrence Berkeley National Laboratory's (LBNL) Demand Response Research Center
(\href
{http://drrc.lbl.gov/publications/demand-response-and-open-automated-demand-response-opportunities-data-centers}{http://drrc.lbl.gov/publications}),
 this paper takes as a starting point LBNL's model for integrating data
centers and the electrical grid. The model describes programs that are used
by the electricity service providers to integrate with their customers (such
as demand response) and methods used to balance the grid supply and demand
of electricity. It also describes strategies that data centers might employ
for managing their electricity and power requirements. This paper tuned this
model's data center strategies for supercomputer centers.  As opposed to data centers, supercomputer centers 
have very high system utilization and are not likely to use virtualization as a strategy.  
Also, supercomputer applications are generally not easily portable between 
geographic locations for a variety of reasons; security, data-locality, system tuning. 
Therefore, although included in the model for data centers, both virtualization and geographic load shifting were eliminated as
potential supercomputing center strategies.


The first section of this paper describes in greater detail the model for
integrating supercomputer centers and the electrical grid. The second
section is a review of prior work on HPC center strategies that might be
deployed for managing electricity and power. In order to further understand
today's relationships, potential partnerships and possible integration
between HPC centers, their electricity providers and the grid, a
questionnaire was deployed whose respondents were Top100 List class
supercomputer centers in the United States. The third section of this paper
describes the results of that questionnaire. The fourth section of the paper
describes opportunities, solutions and barriers. A fifth section describes
conclusions and next steps. Finally, the last section recognizes additional
authors.

