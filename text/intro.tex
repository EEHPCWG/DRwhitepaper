Supercomputer centers with petascale systems for high-performance 
computing (HPC) are realizing the large impact they will be 
putting on their electricity providers with peak power demands of 20MW and instantaneous power fluctuations of 8MW. 
 
The Energy Efficient HPC Working Group 
(\href {http://eehpcwg.lbl.gov/}{EE HPC WG}) 
has been investigating opportunities for large supercomputer sites to more closely 
integrate with their electricity providers. This paper documents the 
results of this investigative activity. 

The EE HPC WG Team took as their starting point a model developed by
Lawrence Berkeley National Laboratory (LBNL's) Demand Response Research Center
that describes challenges and opportunities in which data centers and electricity
service providers may interact and how this integration can advance new market opportunities. \footnote{LBNL
Data Center Grid Integration Activities: http://drrc.lbl.gov/projects/dc} This integration model
describes programs that are used by some of the electricity service providers to encourage particular
responses by their customers and methods used to balance the electric grid supply and demand.
The findings also describe strategies that data centers might employ for utility programs to modify
their electricity and power requirements to lower costs and benefit from utility incentives. The EE HPC WG Team
adopted this model with slight tweaks to reflect the supercomputing environment focus 
(versus the data center as described by LBNL's Demand Response Research Center).

%\href{http://drrc.lbl.gov/publications/demand-response-and-open-automated-demand-response-opportunities-data-centers}
%{http://drrc.lbl.gov/publications}

This paper distinguishes the supercomputer center
as having unique characteristics that distinguish it from the data center. As opposed to data centers, 
supercomputer centers 
have very high system utilization and are not likely to use virtualization as a strategy.  
Also, supercomputer applications are generally not easily portable between 
geographic locations for a variety of reasons; security, data-locality, system tuning. 
Therefore, although included in the LBNL model for data centers, both virtualization and geographic load 
shifting were eliminated as
potential supercomputer center strategies.


Section 2 of this paper describes in greater detail the model for
integrating supercomputer centers and the electrical grid. Section 3 
is a review of prior work on HPC center strategies that might be
deployed for managing electricity and power. In order to further understand
today's relationships, potential partnerships and possible integration
between HPC centers, their electricity providers and the grid, a
questionnaire was deployed whose respondents were Top100 List class
supercomputer centers in the United States. Section 4 of this paper
describes the results of that questionnaire. The fifth section of the paper
describes opportunities, solutions and barriers. A sixth section describes
conclusions and next steps. Finally, the last section recognizes additional
authors.

