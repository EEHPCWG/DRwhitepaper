Supercomputing centers(SCs) with petascale
\footnote{Petascale computing refers to computing systems capable of at
least \(10^{15}\) floating point instructions per second(FLOPS).} 
systems for high-performance computing (HPC) realize the large 
impact they are putting on their electricity service providers (ESPs) with peak 
power demands of 20MW and instantaneous power fluctuations of 8MW. 
Today, with the impetus 
towards Exascale\footnote{Exascale computing refers to computing systems capable of at
least \(10^{18}\)FLOPS.}, 
the need arises to
aid electrical grid reliability  with even larger peak power demands and 
instantaneous power fluctuations.

Electrical grid reliability is extremely important when optimally managing and 
linking electricity supply with demand. Changes in electrical usage by end-users from normal 
consumption patterns can affect the electric supply infrastructure. 
To moderate such effects, it is important to understand how end users like SCs with 
high power demand and instantaneous fluctuations
affect power-supply reliability. 
Consequently, ESPs are now seeking hourly forecasts of power demand from SCs, a day in advance.

The large power demand of SCs also make them pivotal in  
aiding ESPs to maintain reliable supply to other end-users. 
For example, Lawrence Livermore National Laboratory (LLNL)
helps with this challenge by reducing its power usage when temperatures exceed 
100 degrees and residential power usage in the area surges.
This is just one among many other instances where SCs have shown to be  
resourceful in aiding their ESPs.

In the past, large smelters and manufacturing industries extended 
great influence on the electrical grid; but today, because of their increasing power demands,
supercomputing centers are the key players.
While supercomputing centers often focus on energy efficiency to lower costs, 
it is important to understand if this focus is also a key interest for ESPs.
Likewise, it is important for the SC to understand the significance of electrical grid 
reliability---a priority for ESP. 
A mutual understanding of concerns between the SC and ESP can 
produce a symbiotic relation that goes beyond the current producer-consumer 
paradigm, paving the way for possible integration of HPC with the electrical grid.

In this paper, we assume fundamentally that the grid is a given constant. However, 
electrical grid infrastructures will certainly evolve in the 
future \cite{he_architecture_2008}, making grid integration more or less difficult.

The Energy Efficient HPC Working Group (\href {http://eehpcwg.lbl.gov/}{EE HPC WG})  
has been investigating opportunities for large supercomputing sites to  
integrate more closely with their electricity service providers.
The objectives of this investigation are to understand the willingness of SCs to
cooperate with their ESPs, their expectations from their ESPs, and the feasible measures
that SCs could employ to help their ESPs.
To achieve our objectives we deployed a questionnaire to
Top100 List class supercomputer centers in the United States, the results of
which are documented in subsequent sections.

This work leverages prior work on datacenter and grid integration opportunities
done by Lawrence Berkeley National Laboratory's (LBNL) Demand Response 
Research Center \textbf{(cite LBNL DR datacenter integration paper)}
that describes the challenges and opportunities for datacenters and electricity
service providers to interact and how this integration can advance 
new market 
opportunities\footnote{LBNL Data Center Grid Integration Activities: http://drrc.lbl.gov/projects/dc}.
This integration model describes programs that are used by some of the electricity 
service providers to encourage particular responses by their customers and methods 
used to balance the electrical grid supply and demand.
Perhaps one of the most straightforward ways that SCs can begin
the process of participating in demand response is by using this existing software 
infrastructure to manage their electricity requirements in a tightly coupled manner 
with their ESPs, facilitating both energy efficiency and grid reliability.

Section~\ref{sec:supercomputer}  of this paper describes in greater detail the model for
integrating supercomputer centers and the electrical grid. Section~\ref{sec:priorwork}
is a review of prior work on HPC center strategies that might be
deployed for managing electricity and power. Section~\ref{sec:questionnaire} of this paper
describes the results of the questionnaire. Section~\ref{sec:surveysummary}
contains a summary of the questionnaire itself. Section~\ref{sec:opportunities} 
describes opportunities, solutions and barriers. Section~\ref{sec:conclusion} describes
conclusions and next steps. Finally, Section~\ref{sec:additional} recognizes additional
authors.

