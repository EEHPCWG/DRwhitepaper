Supercomputing centers (SCs) with petascale\footnote{Petascale computing refers to computing systems capable of at
least \(10^{15}\) operations or floating point instructions per second (FLOPS).} 
systems for high-performance computing (HPC) can have an outsized 
impact on their Electricity Service Providers (ESPs), with peak 
power demands %replace 'in excess of' with exceeding
exceeding 20 MW and instantaneous power fluctuations of up to 8 MW. 
As the HPC community moves towards exascale computing\footnote{Exascale computing refers to computing systems capable of at
least \(10^{18}\) operations or FLOPS.}, we anticipate that a growing number of facilities will
be reaching or exceeding these service levels, with significant potential 
effect on electrical grid reliability.
In this paper we seek to understand how these anticipated
usage patterns can be integrated safely into the power grid with minimal cost 
and disruption in order to manage
%TP: moderate, mitigate or manage? rephrased this sentence from
%Tp mitigate this risk, in this paper... to In this paper, .... to manage this risk. 
this risk.

%TP: Specify 'electrical' grid as this is unclear
Being a ``good citizen'' on the electrical grid has several historical precedents.
%TP: Change "`For Example"' to "`In the past" because 'For example' is used soon after.
In the past, electrically-intensive industries such as aluminum smelters 
have received preferential pricing in return for predictable loads and
flexibility in reducing power during periods of high consumption.
SCs are already adopting these strategies. 
For example, Lawrence Livermore National Laboratory (LLNL)
reduces its power usage when temperatures exceed 
100 degrees F and the residential power usage in the area surges.
Other %centers TP: replace centers with SCs
SCs are exploring the benefits of predicting hour-ahead and 
day-ahead use in concert with their %providers TP: replace providers with ESPs
ESPs.
A mutual understanding of concerns between SCs and ESPs can 
produce a symbiotic relationship that goes beyond the current producer-consumer 
paradigm, paving the way for possible integration of %HPC TP: replace HPC with SCs
SCs with the electrical grid. Grid Integration in the context of this study refers to the dynamic interaction and value between the demand-side resources (SCs) and the supply-side resources (ESPs) as well as the relationship between the electricity grid and its markets.

The Energy Efficient HPC Working Group (\href {http://eehpcwg.lbl.gov/}{EE HPC WG}) 
investigates opportunities for large supercomputing sites to 
integrate more closely with their ESPs.
We seek to understand the willingness of SCs to
cooperate with their ESPs, their expectations from their ESPs, and the feasible measures
that SCs could employ to help their ESPs.
To achieve our objectives we developed a questionnaire and distributed it to 
the Top100 SCs in the United States. 

This paper %TP replace 'work' with 'paper' 
leverages prior work on datacenter and grid integration opportunities
done by Lawrence Berkeley National Laboratory's (LBNL) Demand Response 
Research Center \cite{LBNL}. This prior work describes the challenges and opportunities for datacenters and %electricity service providers 
ESPs to interact with each other and how this integration can advance 
new market opportunities \cite{Ghatikar2012a, Ghatikar2012b}. 
%\textbf{(cite LBNL DR datacenter integration paper)}
% Now a reference. TP. \footnote{LBNL Data Center Grid Integration Activities: http://drrc.lbl.gov/projects/dc}.
This integration model describes programs that are used by some of the ESPs %electricity service providers 
to encourage particular responses by their customers and methods 
used to balance the electrical grid supply and demand. This is referred to as \emph{Demand Response (DR)}.
 %TP: This defines demand response, I'm clarifying this here.
%Perhaps TP: No "perhaps".

%Moved this after, as DR was not defined and was being used in the bullet #1,3.
Eleven sites responded to the aforementioned questionnaire. Based on these responses, we noted a few
primary observations:
\begin{itemize}
\item Only 20\% of SCs currently communicate with their ESPs about DR issues.
\item SC managers believe that the candidate solutions most likely to be effective for
responding to ESP requests involve coarse-grained power management techniques, job
scheduling techniques, and shutting down computing resources.
\item A stronger relationship, including DR capabilities, between SCs and
ESPs can lead to both energy savings and cost savings over time, and in some
cases such capabilities might become a requirement for large SCs located in energy-challenged
locations.
\end{itemize}

%TP: Don't need this here as this section ends with what each section presents.
%Section~\ref{sec:questionnaire} provides an in-depth analysis of the responses to the
%questionnaire from which these conclusions were drawn.

One of the most straightforward ways that SCs can begin
the process of engaging %replace participating with engaging, as participating is used twice in the sentence
in integration is by participating in efforts to
develop software 
infrastructure to manage their electricity requirements in a tightly coupled manner 
with their ESPs, facilitating both energy efficiency and grid reliability. In addition, this will provide for extensive funding and cost analysis and help the community base future requirements for SCs and ESPs on facts and a proven set of measurements.

Our analysis in this paper focuses %TP:not focused, but focuses. 
on SCs in the United States. However, 
the findings can be extended to and may relate to SCs in other countries with similar practices. 
Electric grid infrastructure and market design are highly dependent on %on, not upon. TP.
governmental regulations that vary across geographies. We restricted the initial analysis to the understanding of 
electricity markets in the United States. Future work can extend the analysis to the electricity markets in Europe and other geographies.

The paper is organized as follows.
Sections~\ref{sec:ESPintegration} and~\ref{sec:supercomputercenters} of this paper
describe in greater detail the model for 
integrating SCs and the electrical grid.
Section~\ref{sub:priorwork}
reviews prior work in SC strategies. %TP: HPC replaced with supercomputing
Section~\ref{sec:questionnaire} provides the results of the questionnaire. 
Section~\ref{sec:opportunities} 
discusses the several opportunities, solutions and barriers that have been highlighted
by the survey results. We offer our conclusions in Section~\ref{sec:conclusion} along with our plan for future work. Additional authors are listed in Section~\ref{sec:additional} and the \nameref{Appendix} summarizes the survey questions.
